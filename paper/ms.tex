% upon AAS submission
%\documentclass[12pt,twocolumn,tighten,linenumbers]{aastex63}
%\documentclass[12pt,twocolumn,tighten,linenumbers,trackchanges]{aastex63}
% drafting / arxiv
\documentclass[11pt,twocolumn,tighten]{aastex63}
\turnoffedit

\usepackage{apjfonts}
\usepackage{url}
\usepackage{hyperref}
\usepackage{natbib}
\usepackage{amsmath,amstext,amssymb}
\usepackage{xcolor, fontawesome}
\usepackage{color}

\newcommand{\rprs}{{$R_p/R_{\star}$}}
\newcommand{\vsini}{{$V \sin i$}}
\newcommand{\teff}{T$_{\textrm{eff}}$}
\newcommand{\kms}{{km\,s$^{-1}$}}
\newcommand{\gcc}{{g\,cm$^{-3}$}}
\newcommand{\rstar}{{$R_\star$}}
\newcommand{\rhostar}{{$\rho_\star$}}
\newcommand{\mearth}{{M$_\oplus$}}
\newcommand{\rearth}{{R$_\oplus$}}
\newcommand{\rsun}{{R$_\odot$}}
\newcommand{\msun}{{M$_\odot$}}

\newcommand{\minus}{\scalebox{0.5}[1.0]{$-$}}

\newcommand{\eg}{e.g.,} 

\newcommand{\mstype}{letter}

%%%%%%%%%%%%%%%%
% INSTITUTIONS %
%%%%%%%%%%%%%%%%
\newcommand{\caltech}{Department of Astronomy, MC 249-17, California Institute of Technology, Pasadena, CA 91125, USA}
\newcommand{\mitkavli}{MIT Kavli Institute and Department of Physics, 77 Massachusetts Avenue, Cambridge, MA 02139}
\newcommand{\berkeley}{Astronomy Department, University of California, Berkeley, CA 94720, USA}

%%%%%%%%%%%%%%%%%%%%%%%%%%%%%%%%%%%%%%%%%%
%%%%%%%%%%%%%%%%%%%%%%%%%%%%%%%%%%%%%%%%%%
%%%%%%%%%%%%%%%%%%%%%%%%%%%%%%%%%%%%%%%%%%

%
% ms specific numbers
%

%%%%%%%%%
% STARS %
%%%%%%%%%
% number of stars monitored by Kepler (quoting Santos19/21)
\newcommand{\nkeplerstars}{$\approx$160{,}000}
% fraction of kepler stars w/ rotation periods
\newcommand{\nstarswithprot}{{$\sim$50{,}000}}
% fraction of stars with rotation periods, with teff/logg in Berger+20
\newcommand{\fracstarswithprotwithbtwenty}{{$\sim$94\%}}
% complement of above
\newcommand{\fracstarswithprotwithoutbtwenty}{{$\sim$6\%}}

%%%%%%%%%%%
% PLANETS %
%%%%%%%%%%%
% number of KOIs in Q1-Q17 DR25 table
\newcommand{\nkois}{{$\sim$8{,}000}} %FIXME precise
% number of KOIs in Q1-Q17 DR25 table that are not known false positives
\newcommand{\nkoisnofp}{{$\sim$4{,}000}} %FIXME precise
% number of KOIs in Q1-Q17 DR25 table that have rotation periods
\newcommand{\nkoiswithprot}{{$\sim$2{,}000}}
% number of KOIs in Q1-Q17 DR25 table that are not known false positives that have rotation periods
\newcommand{\nkoisnofpwithprot}{{$\sim$1{,}000}}
% fraction of KOIs that are not FPs with B+20 parameters
\newcommand{\frackoisnofpwithprotwithbtwenty}{{$\sim$92\%}}

%%%%%%%%%%%
% LITHIUM %
%%%%%%%%%%%
% number of KOIs that are not known FPs that have a HIRES spectrum
\newcommand{\nkoiswithhires}{{1{,}500}}
% number of KOIs for which this work and Berger2018 overlap
\newcommand{\nbergeroverlap}{{500}}

%%%%%%%%
% AGES %
%%%%%%%%
\newcommand{\nkepstarswithages}{NNN}
\newcommand{\nkoisnofpwithages}{NNN}
\newcommand{\nkoisgoldwithages}{NNN}

%%%%%%%%%%%%%
% OFT-CITED %
%%%%%%%%%%%%%
\defcitealias{McQuillan_2014}{M14}
\defcitealias{Mazeh_2015}{M15}
\defcitealias{Santos_2019}{S19}
\defcitealias{Santos_2021}{S21}

%%%%%%%%%%%%%%%%%%%%%%%%%%%%%%%%%%%%%%%%%%
%%%%%%%%%%%%%%%%%%%%%%%%%%%%%%%%%%%%%%%%%%
%%%%%%%%%%%%%%%%%%%%%%%%%%%%%%%%%%%%%%%%%%

\begin{document}

\title{Gyrochronal and Lithium Ages for Stars and Planets Observed by Kepler}

\correspondingauthor{Luke G. Bouma}
\email{luke@astro.caltech.edu}

\received{---}
\revised{---}
\accepted{---}
\shorttitle{gyro-johannes} 

\shortauthors{Bouma et al.}

\author[0000-0002-0514-5538]{Luke~G.~Bouma}
\altaffiliation{51 Pegasi b Fellow}
\affiliation{\caltech}

\author{Authors to include (ORDER AFTER LGB NOT YET ASSIGNED MEANING)}

\author[0000-0001-7967-1795]{Elsa~K.~Palumbo}
\affiliation{\caltech}

\author{Lynne~A.~Hillenbrand}
\affiliation{\caltech}

%
% HIRES Collaborators
%
\author[0000-0002-0531-1073]{H. Isaacson}
\affiliation{\berkeley}
%
\author[0000-0001-8638-0320]{A. W. Howard}
\affiliation{\caltech}
%
\author{HIRES observers to be named}
\affiliation{Various}

% <250 words
\begin{abstract}
  We report rotation- and lithium-based ages for stars and planets observed by
  the primary Kepler mission.  Stellar rotation periods are collected from
  previous analyses of the Kepler light curves; lithium content is assessed
  using HIRES spectra.
  We report ages for \nkoisgoldwithages\ FGK stars with transiting
  planets, with relative statistical precisions $\sigma_t / t$ of XX
  +YY -ZZ\%; a factor of X.X +Y.Y -Z.Z more precise than
  state-of-the-art isochronal ages for the same stars.
  The results include XXX planets younger than 1\,Gyr (at 2$\sigma$),
  and YYY planets younger than 0.5\,Gyr (at 2$\sigma$).
  In the base sample of dwarf FGK stars observed by Kepler, we find
  that the intrinsic age distribution is inconsistent with a uniform
  distribution, and instead rises roughly linearly from 0-3\,Gyr;
  stars 2\,Gyr old are twice are common in the Kepler field as stars
  1\,Gyr old.
  This agrees with results from previous isochronal analyses, and extends
  beyond their age sensitivity, down to $\approx$0.1\,Gyr.
  %TODO check A-D test
  The sample of Kepler planets has a (in)consistent age distribution.
  Analyzing the completeness of the Kepler
  detection pipeline, we interpret the majority of the change to be
  caused by selection effects (or astrophysics).  We also comment on a
  few previously reported evolutionary trends in the size evolution of
  close-in mini-Neptunes (2-4\,\rearth) and super-Earths
  (1-2\,\rearth), a few of which we recover at comparable
  ($\sim$3$\sigma$) significance to previous work, and one of which
  we note for the first time.
\end{abstract}

\keywords{Exoplanet evolution (491), Stellar ages (1581)}

\section{Introduction}
\label{sec:intro}

Discovering a young planet requires solving two problems: find the
planet, and measure the star's age.  Each problem can be
solved in different ways, yielding many possible permutations for
discovery.  In this manuscript, we will focus on the transit method
for planet discovery, and rotation and lithium-based methods for stellar age
measurement.  Rotation-based age-dating, also known as
gyrochronology, will provide the most information for the largest
number of stars.  The lithium-based technique will provide an
important sanity check for the subset of stars younger than a few
hundred million years.

To help build intuition, it is useful to assume that the age
distribution of stars within the nearest few hundred parsecs of the
Sun (i.e.~in the Solar Neighborhood) is uniform between 0 and 10\,Gyr
\citep{Nordstrom_2004}.  We will revisit this assumption later in the
manuscript.  While the youngest 0.1\% of stars ($\lesssim$10\,million
years; Myr) are perhaps the most constraining for studies of planet
formation, they are extremely challenging targets for close-in
exoplanet detection 
\citep[see e.g.][]{Damasso_2020,Bouma_2020_ptfo,Donati_2020}.  Studies of
the early orbital, structural, and atmospheric evolution of exoplanets
are perhaps most technically feasible for the $\approx$1\% of stars
with ages between 10 and 100\,Myr, or perhaps even the 10\% of stars
younger than 1\,Gyr.  However, since stellar ages are roughly
uniformly distributed in the solar neighborhood, there is a limit on
the sample size of young planets that will be detectable in these age
regimes, particularly relative to their older breathren.

%TODO mention the RV approach?  at least the Hyades etc work...
A wide range of approaches have, over the past ten years, begun to
establish the subfield of young, close-in planet discovery.
Pioneering discoveries with the prime Kepler mission (hereafter,
``Kepler'') focused on stars with known ages---stars in clusters---and
searched them for transiting planets \citep{Meibom_2013}.  Planets
discovered in this way are assumed to have the same age as the host
cluster, which are the standards for the current astrophysical age
scale.  The resulting ages are precise at the $\approx$10\% level.  K2
and TESS have yielded larger numbers of young planets in clusters than
Kepler, because each of these missions has observed a larger fraction
of the sky, and therefore more stars with known ages
\citep[e.g.][]{Mann_K2_25_2016,Mann_2017,Curtis_2018,Livingston_2018,David_2019,Bouma_2020_toi837,Rizzuto_2020,Plavchan_2020,Newton_2021,Nardiello_2022,Tofflemire_2021,Zhou_2022,Zakhozhay_2022,Wood_2023}.

An important limitation of the ``cluster-first'' approach is that only
$\approx$1\% of stars within the nearest few hundred parsecs can
currently be associated with their birth cluster
\citep[e.g.][]{Zari_2018,CantatGaudin_2020,Kounkel_2020,Kerr_2021}.
This means that there are many ``young'' 0.05-1\,Gyr stars in the
field that were not part of the selection function of these searches.
Roughly 40 transiting planets are currently known to be younger than
1\,Gyr, with an age precision $t/\sigma_t > 3$ (see
Figure~\ref{fig:rp_period_age}).  However, Kepler discovered
$\approx$4{,}000 planets; and so barring a major discovery bias
against young planets, we would expect a significant portion of the
Kepler planets---perhaps 10\%, or 400 planets---to be younger than
1\,Gyr.  If one is willing to compromise on age precision, those
planets might be identifiable.

\begin{figure}[!t]
	\begin{center}
		\leavevmode
		\includegraphics[width=0.49\textwidth]{f1.pdf}
	\end{center}
	\vspace{-0.6cm}
	\caption{
		{\bf Radii, orbital periods, and ages of transiting exoplanets}.
    Planets younger than 1\,Gyr with ages more precise than a
    factor of three are emphasized. 
    ({\bf cite relevant papers?})
    Parameters are from the \citet{PSCompPars}.
		\label{fig:rp_period_age}
	}
\end{figure}

Have these young Kepler planets already been found?  If they have, the
most likely studies to have found them would be those by
\citet{Berger_2020b_rpage}, \citet{David_2021} or
\citet{Petigura_2022}.  The Berger and Petigura studies focused on
isochronal age-dating of individual stars, which is most precise in
the context of Kepler for stars that are beginning to evolve off the
main-sequence.  The \citet{David_2021} study leveraged both isochronal
ages from an earlier analogous study
\citep{Fulton_Petigura_2018_cks_vii}, and (in their core analysis) a
relative gyrochronal age-ranking approach that sorted stars based on
whether they rotated faster or slower than a gyrochrone age of
$\approx$2.6\,Gyr \citep{Meibom_2015,Curtis_2020}.

Theoretical predictions for planet evolution in the Kepler sample are
most important in the context of mini-Neptunes and super-Earths, since
these are the most abundant planets.  Pertinent predictions include:
{\it a)} that mini-Neptunes should cool and contract, roughly
following a $R_p \propto t^{-0.1}$ scaling \citep{Gupta_2019}; {\it
b)} that large and close-in mini-Neptunes at the edge of the
photoevaporation desert should shrink over the atmospheric loss
timescale, $t_{\rm loss}$ \citep{Owen_Lai_2018}; {\it c)} that the
abundance of 4-8\,R$_\oplus$ planets with masses of 5-30\,M$_\oplus$
should decrease over $t_{\rm loss}$---the upper-envelope of the
mini-Neptune size distribution should corresponding decrease,
e.g.~from $\approx$4.5\,R$_\oplus$ at $\approx$200\,Myr to
$\approx$3.5\,R$_\oplus$ at $\approx$700\,Myr
\citep[e.g.][]{Rogers_2021}; {\it d)} that the super-Earth to
mini-Neptune ratio, $\theta$, should increase over $t_{\rm loss}$
\citep[e.g.][]{Rogers_2021}, with the mini-Neptune abundance
decreasing by at least a factor of two or three, and the super-Earth
abundance correspondingly increasing {\bf BY HOW MUCH?}); {\it e)} that the
upper boundary of the super-Earth population should increase over the
same timescale as $\theta$ is increasing, since more massive
super-Earth cores should be able to retain their envelopes for longer
than less massive cores.

The details of the predictions are flexible, in that they depend on
the distributions of planetary core masses, initial atmospheric mass
fractions, and initial entropies of the atmospheres after disk
dispersal.  For instance, different sources of heating---from the star
\citep{Owen_Wu_2013,Lopez_Fortney_2014,Jin_2014}, planetary interior
\citep{Gupta_2019}, or even giant impacts
\citep{Biersteker_Schlichting_2019}---can produce very different
histories for the size evolution of planets at these early times
\citep[e.g.][]{Owen_2020}.  The generic prediction that the ``upper
envelope'' of the mini-Neptune size distribution should shrink due to
Kelvin-Helmholtz cooling is likely hard to avoid; predictions however
concerning the ``radius valley'' and movement of planets across it are
dependent on the timescales and prevalence of atmospheric mass loss.
In specific scenarios for initial atmospheric accretion, the radius
gap might form without needing to invoke mass loss at all
\citep{Lee_2022}, in which case there could be very little evolution
of the planetary size distribution between 1 and 4\,\rearth\ over the
first gigayear.

Can stellar age measurements provide lines of evidence for or against
the different models?  Neither {\it (a)} nor {\it (b)} has yet had a
convincing---or even suggestive--- detection reported (see
\citealt{Petigura_2022}).  While the apparent over-abundance of large
very young planets ({\it (c)}) has been noted by multiple authors
\citep[e.g.][]{Bouma_2020,Mann_2022}, its presence is almost certainly
exacerbated by a selection effect against sub-Neptune sized planet
detection at $<$100\,Myr \citep{Zhou_2021,Bouma_2022}.  And finally,
{\it (d)} and {\it (e)} have been explored by
\citet{Berger_2020b_rpage} and the twin \citet{David_2021} and
\citet{Sandoval_2021} studies.  Both \citet{Berger_2020b_rpage} and
\citet{Sandoval_2021} reported that $\theta$ increases with time, at a
statistical significance of $\approx$2-3$\sigma$.  For
\citet{Berger_2020b_rpage}, this conclusion was reached by comparing
the sizes of 85 planets with median isochronal ages younger than
1\,Gyr with a property-matched sample of planets older than 1\,Gyr.
In \citet{Sandoval_2021}, the conclusion was reached by computing the
ratio while performing a Monte Carlo re-sampling procedure from the
isochronal ages and planetary sizes computed by
\citet{Fulton_Petigura_2018_cks_vii}.  Finally, effect {\it (e)} was
reported by \citet{David_2021}, who found that the average location of
the radius valley shifted from $\approx$1.8\,R$_\oplus$ to
$\approx$2.0\,R$_\oplus$ between 1.8 and 3.2\,Gyr.

The fact that stellar ages---and especially isochronal ages---are
correlated with stellar masses and metallicities is a source of stress
for many in this subfield.  \citet{Sandoval_2021} for instance noted a
strong trend in $\theta$ as a function of stellar metallicity in
addition to the trend they noted in age, and cautioned that this could
also explain their observations.  Independently, \citet{Petigura_2018}
and \citet{Petigura_2022} have reported trends between small-planet
occurrence and both stellar metallicity and mass; these parameters
seem, at first glance, to be more important predictors of bulk changes
in the planetary population than age.  Given these known correlations,
it would be interesting to consider population models in which
variables such as metallicity, mass, and age were added in succession
\citep[e.g.][]{Thorngren_2021}.  For such an analysis to be effective,
the measurement precision of these different parameters would ideally
be as high as possible.

Gyrochronology offers ages precise to $\lesssim$30\% for FGK stars
between 0.5-2.5\,Gyr, and can also provide useful age limits at
earlier times \citep{Bouma_2023}.  The idea of using a star's
spin-down as a clock is quite old
\citep{Skumanich_1972,Noyes_1984,Kawaler_1989,Barnes03,Mamajek_2008,Angus_2015},
and physics-based models for the spin-down itself can clarify many
aspects of how the stellar winds, internal structure, and magnetic
dynamo all evolve
\citep[e.g.][]{Matt_2015,Gallet_Bouvier_2015,Spada_2020}.  Key
for the present analysis is the fact that we can empirically calibrate
rotation-based ages using measured sequences of rotation periods
in open clusters
(\citealt{Bouma_2023}, and references therein).  This approach
ultimately ties the rotation age scale to the cluster age scale, and
is limited in precision by the intrisic scatter in the $P_{\rm rot}$
sequences, and in accuracy by systematic uncertainties in the evolving
spin-down rates, and in the cluster ages themselves.

Lithium-based age-dating includes two qualitatively distinct regimes:
depletion boundary ages derived for M-dwarfs in star clusters, and the
less precise decline-based ages for individual field FGK stars
\citep{Soderblom_2010}.  Here we focus on the latter approach, which
relies on the gradual empirical observation that the Li abundances of
FGK stars decline as a function of time, for reasons different from
the standard convective mixing and burning scenario (CITE, CITE).
Empirically, this decline has recently been documented by
\citet{Jeffries_2023}, who built a model for how the equivalent width
(EW) of the \ion{Li}{1} 6708\,\AA\ doublet evolves as a function of
stellar effective temperature and age for a set of 6{,}200 stars in 52
open clusters.  Two-sided lithium ages are useful for Kepler (FGK)
stars between $\approx$0.03-0.5\,Gyr, with a strong dependence on
spectral type since the K-dwarfs lose their surface lithium much
faster than G-dwarfs.  The precision of the ages reported by the
\citet{Jeffries_2023} method in this regime are in the range of
0.3-0.5~dex.

The methods by which we select the stellar and planet samples are 
discussed in Section~\ref{sec:selection}.
The age-dating methods are described and validated in
Section~\ref{sec:agemethod}.
The population-level trends for both the parent stellar sample and
the planet sample are discussed in Section~\ref{sec:results}.
A few conclusions are offered in Section~\ref{sec:conclusions}.


\section{Sample Selection}
\label{sec:selection}

This work is focused on stars surveyed by Kepler for which ages can be
inferred using either stellar rotation, lithium, or both.  These
``age-dateable stars'', $\mathcal{S}$, are a relatively small subset
of the \nkeplerstars\ Kepler targets.  Rotation periods provide
the majority of the information, since they have been reported for
$\approx$55{,}000 Kepler stars
\citep[e.g.][]{McQuillan_2014,Santos_2021}.  Measurements of the
\ion{Li}{1} 6708\,\AA\ doublet require high-resolution
($R\gtrsim$10{,}000) spectra, and are only available for a few
thousand Kepler objects of interest (KOIs).  We therefore choose to
define our selection function as Kepler stars with measured
rotation periods ($\mathcal{S}$; Section~\ref{subsec:starsel}), and
Kepler planets whose host stars have measured rotation periods
($\mathcal{P}$; Section~\ref{subsec:planetsel}).


\subsection{Kepler stars with rotation periods}
\label{subsec:starsel}

We select stars with rotation periods as determined by previous
investigators.  Studies of stellar rotation across the entire Kepler field
include those by \citet{McQuillan_2014} (\citetalias{McQuillan_2014}),
\citet{Reinhold_2015},
\citet{Santos_2019} (\citetalias{Santos_2019}), and
\citet{Santos_2021} (\citetalias{Santos_2021}).
\citetalias{McQuillan_2014} used an approach
based on the autocorrelation function to detect 34{,}030 rotation
periods for main-sequence Kepler targets cooler than 6500\,K, and excluded
known eclipsing binaries and KOIs.  
\citet{Reinhold_2015} used an iterative Lomb-Scargle based approach to
analyze $\approx$40{,}000 stars with a variability range $R_{\rm
	var}>0.3$\% that were not known EBs, planet candidates, or pulsators,
and reported primary rotation periods for 24{,}124 of these stars.
\citet{Reinhold_2015} also reported secondary periods for two thirds
of the stars with primary periods, which could be caused by
differential rotation or finite spot lifetimes.  
Finally,
\citetalias{Santos_2019} and
\citetalias{Santos_2021} combined a wavelet analysis and
autocorrelation-based approach, and cumulatively reported rotation
periods for 55{,}232 main-sequence and subgiant FGKM stars.
\citetalias{Santos_2019} and \citetalias{Santos_2021} included known
KOIs and binaries, and assigned them specific quality flags.  
The rotation periods of KOIs have received considerable additional
scrutiny
\citep[e.g.][]{Walkowicz_2013,Mazeh_2015,Angus_2018,David_2021}.

%After reviewing the relevant literature and considering the nature of
%the rotation period detection problem,
We chose to adopt the results of \citetalias{Santos_2019} and
\citetalias{Santos_2021} as our default rotation periods.
\citetalias{Santos_2021} provides a detailed comparison against
\citetalias{McQuillan_2014}; the brief summary is that the periods
agree for 99.0\% of the 31{,}038 period detections in common between
the two studies.  \citetalias{Santos_2021} classified the 2{,}992
remaining stars from \citetalias{McQuillan_2014} as ``non-periodic''
based on updated knowledge of contaminants (e.g.~giant stars and
eclipsing binaries) and visual inspection.  In addition,
\citetalias{Santos_2021} report rotation periods for 24{,}182
main-sequence and subgiant FGKM stars that were not reported as
periodic by \citetalias{McQuillan_2014}.  Our interpretation is that
these detections are likely more challenging than those from
\citetalias{McQuillan_2014}, perhaps due to a lower variability
amplitude, or a longer rotation period.  A relevant flag for stars
missing \citetalias{McQuillan_2014} periods is included in
Table~\ref{tab:stars}.

Concatenating the results of \citetalias{Santos_2019} and
\citetalias{Santos_2021} yields \nstarswithprot\ Kepler stars with
rotation periods.


%FIXME

\subsection{Kepler objects of interest}
\label{subsec:planetsel}

We considered two different samples of planets: those in the fully
automated Q1-Q17 DR25 KOI Table \citep{Thompson_2018}, which is
suitable for planet occurrence rate calculations, and also those in
the cumulative KOI table, which included the best knowledge available
on any given planet candidate as of 2023 Jun 6, but incorporated
human-based vetting \footnote{See
\url{https://exoplanetarchive.ipac.caltech.edu/docs/PurposeOfKOITable.html}}.
For the main body of this work, we will refer to the former
homogeneous sample, which includes \nkois\ KOIs, \nkoisnofp\ that are
not known false positives;
Appendix~\ref{app:cumulativekoi} highlights some additional young
planets that are omitted in this approach.

Crossmatching the Q1-Q17 DR25 KOI table against our stars with
rotation periods yields \nkoiswithprot\ KOIs with rotation periods, of which \nkoisnofpwithprot\ 
are not known false postives.
These objects will comprise the core data for our gyrochronology analysis.

As a sanity check on the statistical uncertainties of these rotation
periods, we compared our adopted \citetalias{Santos_2019} and
\citetalias{Santos_2021} periods with those reported by
\citet{McQuillan_2014} and \citet{Mazeh_2015}.
We found that for $P_{\rm rot}\lesssim$15\,days, the two 
datasets at a precision of $\lesssim$0.01$P_{\rm rot}$.  At longer periods of 
$P_{\rm rot}\approx$30\,days, the agreement was typically at the
$\lesssim$0.03$P_{\rm rot}$ level, and
the upper envelope of the period difference tended to increase
linearly with period.
Based on this comparison, we adopted a simple prescription for the
period uncertainties,
such that there are 1\% relative uncertainties below $P_{\rm
rot}=15$\,days, and a linear increase thereafter, with slope set to
guarantee 3\% $P_{\rm rot}$ uncertainties at 30 days.



\subsection{Lithium sample}
A final subcomponent of our analysis involves an age assessment based on
the equivalent width of the \ion{Li}{1} 6708\,\AA\ doublet.  For this
component of the work, we primarily rely on spectra acquired with the
High Resolution Echelle Spectrometer (HIRES;
\citealt{vogt_hires_1994}) on the Keck I 10m telescope.
The majority of these spectra are archival, and were acquired
as part of the California Keck Survey (CKS).
A subset were acquired by our team in our own observing programs.
%TODO quantify.
%TODO compare to berger2018?

These HIRES spectra are available for \nkoiswithhires\ of the
\nkoisnofp\ non-false-positive KOIs.
In the broader stellar sample, they are available for a much smaller 
fraction of the stars observed by Kepler, because they were generally only
acquired due to the presence of a transiting planet candidate.
Given this selection function,
we leverage these spectra in our analysis as a way to scrutinize the
ages of just the planet sample.



\section{Stellar Properties}

% https://gea.esac.esa.int/archive/documentation/GDR3/Data_analysis/chap_cu8par/sec_cu8par_apsis/ssec_cu8par_apsis_gspphot.html

As our default source for stellar temperatures and surface gravities, we adopt
the values from the Gaia-Kepler Stellar Properties Catalog (GKSP;
\citealt{Berger_2020a_catalog}).
The GKSP parameters are available for stars with ``AAA'' 2MASS photometry,
measured parallaxes in Gaia DR2,  and g-band photometry from either the KIC or
the Kepler-INT survey.
The parameters themselves were derived using \texttt{isoclassify} (CITE
Huber2017) to interpolate over the MIST isochrone grids (CITE Choi2016,
Dotter2016), given the SDSS $g$ and 2MASS $K_{\rm s}$ photometry, the Gaia DR2
parallexes, and $V$-band extinction from the (CITE) Green2019 reddening map.
The resulting stellar parameters are available for
\fracstarswithprotwithbtwenty\ of the \nstarswithprot\ Kepler stars
with rotation periods.

%For the remaining \fracstarswithprotwithoutbtwenty\ of stars that lack
%temperatures and surface gravities, we adopt the values 
%from the Gaia DR3 General Stellar Parametrizer from Photometry (GSP-Phot; CITE).
%GSP-Phot uses the low-resolution Gaia BP and RP spectra, mean $G$ magnitudes,
%and parallaxes, in concert with the PARSEC isochrones (CITE Tang2014, Chen2015,
%Pastorelli2020) to infer the stellar effective temperatures, surface gravities,
%metallicities, and line-of-sight extinctions.
%In the planet sample,
%\frackoisnofpwithprotwithbtwenty\ of the
%the non-false-positive KOIs with rotation periods have parameters from
%\citet{Berger_2020a_catalog}.

For the remaining \fracstarswithprotwithoutbtwenty\ of stars that lack
temperatures and surface gravities, we adopt the values reported by
\citet{Santos_2019} and \citet{Santos_2021}, which for these cases
primarily derive from the \citet{Mathur_2017} DR25 Kepler Stellar
Properties Catalog, and are mostly derived from photometry.  In the
planet sample, \frackoisnofpwithprotwithbtwenty\ of the the
non-false-positive KOIs with rotation periods have parameters from
\citet{Berger_2020a_catalog}, and the remainder similarly are drawn
from DR25. 

As a sanity check on our effective temperature scale, we then used the 
the CITE Green2019 reddening map, combined with the Gaia DR2 $G_{\rm
BP}-G_{\rm RP}$ color, to calculate effective temperatures using the
relation suggested by \citet{Curtis_2020}.
This latter effective temperature scale was demonstrated to have a 
scatter of $\approx$50\,K for stars near the main-sequence.
We find {\bf X, Y, Z.}
We ultimately favor the \citet{Berger_2020a_catalog} scale
because it uses 2MASS $K_{\rm s}$ photometry, and is tied to an actual
isochronal fit.

%TODO: plot uncertainties in teff/logg for both parameter sets.

% TODO: add that plot showing differences btwn the two teff scales?


\section{Age Measurement}
\label{sec:agemethod}

\subsection{Gyrochrone ages}
To calculate gyrochrone ages, we use \texttt{gyro-interp}
\citep{Bouma_2023}.  This framework was built to statistically address
the issue that, especially over the first gigayear of stellar
spin-down, stars with the same mass and same rotation period can have
very different ages \citep[e.g.][]{Curtis_2019_ngc6811}.  The
gyrochrone age posterior should therefore incorporate this intrinsic
population-level scatter into the precision with which the age can be
measured.  The three main assumptions behind the \texttt{gyro-interp}
framework are as follows.  {\it 1)} The stars are on the main
sequence, {\it 2)} they do not have any binary companions,
and {\it 3)} the rotation period distribution
for field stars of a given mass and age is identical to that of
the calibration open clusters (metallicity differences, for instance, are
ignored).  
Regarding the former two assumptions:
pre-main-sequence and post-main-sequence evolution introduce
variations in the stellar angular momentum budget.  Similarly, tidal
effects might systematically spin up (or down) stars in binaries.

Using our adopted effective temperatures and rotation periods, we ran 
\texttt{gyro-interp} over an age grid of 0 to 5\,Gyr, linearly spaced
over 500 grid points.
\texttt{gyro-interp} itself is only calibrated out to 4\,Gyr, the
adopted age of the M67 cluster \citep[see][]{Dungee_2022,Gruner_2023}.
We used the spin-down law between the reference open clusters by
interpolating using piecewise cubic hermite interpolating polynomials,
and integrated Equation~1 of \citep{Bouma_2023} at the default grid
resolution.
This yielded age posterior probabilities over our requested age grid
for \nkepstarswithages\ Kepler stars, and \nkoisnofpwithages\
non-false-positive KOIs.


\subsection{Lithium ages}
To measure the lithium equivalent widths for the \nkoiswithhires\ KOIs
with HIRES spectra, we use the procedure described by
\citet{Bouma_2021}.
The stars of interest for this work are FGK stars, and so the
continuum in the vicinity of the
\ion{Li}{1} 6708\,\AA\ feature is well-defined.
We therefore numerically integrate a single best-fit Gaussian over a
local window, and estimate the uncertainties on the line width through
a Monte Carlo procedure that bootstraps against the local scatter in the
continuum.
Generally speaking, this approach does not correct for the neighboring
\ion{Fe}{1} 6707.44\,\AA\ blend.

To evaluate the accuracy and precision of this approach, after
applying an initial iteration of this method on the \nkoiswithhires\
KOIs with HIRES spectra, we compared our lithium equivalent widths
with those reported by \citet{Berger_2018}.
For the \nbergeroverlap\ stars in both samples, 
we found {\bf X, Y, Z.}

To calculate lithium ages, we then use EAGLES \citep{Jeffries_2023}.
{\bf TODO: explain why}
%TODO: explain why.



\subsection{Quality flags}
In our core gyrochrone analysis, we calculate gyrochrone age
posteriors for all the stars with reported rotation periods and
effective temperatures.  To select the stars for which we expect the
gyrochronal age to be valid, we then build a set of quality flags, and
apply them in the sub-analyses that follow.  We defined these filters
as follows, each of which is included in Tables~\ref{tab:stars}
and~\ref{tab:planets}.

{\it Subgiants}---We flagged stars as possible subgiants if they had
$\log g < 4.2$.  We set this threshold by examining the
relevant flag from \citet{berger_2018_radii_evolnstates},
cross-matched against \citet{Berger_2020a_catalog}.
\citet{berger_2018_radii_evolnstates} performed this classification
using the PARSEC evolutionary models, stellar radii, and DR25 stellar
properties catalog effective temperatures \citep{Mathur_2017}.

{\it Photometric binaries}---We plotted color--absolute magnitude
diagrams using the Gaia DR2 photometry, in $M_{\rm G}$ vs. $(G_{\rm
BP} - G_{\rm RP})_0$, and $(G - G_{\rm RP})_0$.  We then manually drew
loci to flag over or under-luminous stars in each diagram, typically
more than $\approx$0.5 magnitudes above or below the main sequence.
%TODO refer to a figure!!!

{\it RUWE}---Following recommendations in the Gaia DR3
documentation\footnote{\url{https://gea.esac.esa.int/archive/documentation/GDR2/Gaia_archive/chap_datamodel/sec_dm_main_tables/ssec_dm_ruwe.html}},
we flagged stars with a Gaia DR3 renormalized unit weight error (RUWE)
exceeding 1.4 as possible binaries.  Such astrometric outliers can be
either bonafide astrometric binaries, or more often are marginally
resolved point sources for which the single-source PSF model used in
the Gaia pipeline induces spurious astrometric scatter.

{\it Crowding}---We searched the Gaia DR3 point source catalog for
stars within 1 Kepler pixel (4$''$) of every target star.
While many such companions are not physically associated with the
target star, their presence can bias stellar parameters and stellar
rotation period measurements.
We therefore flag any stars with neighbors down to 10$\times$ the 
brightness of the target star within this region ($\Delta G < 2.5$).

{\it Gaia DR3 Non-Single-Stars}---The Gaia DR3
\texttt{non\_single\_star} column in the \texttt{gaia\_source} table
flags known eclipsing, astrometric, and spectroscopic binaries.  We
directly include this column.

{\it Final calibration sample}---{\bf todo: mention any extra flags
from Berger2020, or perhaps in the KOI catalog itself, or perhaps the
amplitude cuts you might want to apply to the rotation period catalog}
%The combination of the filters
%described above yields the set of stars that show no evidence for
%binarity or crowding.  However, some of the rotation period analyses
%in Table~\ref{tab:clusters} include additional relevant quality flags.
%For instance, light curves showing multiple photometric periods can
%indicate unresolved binarity.  We used all relevant filters available
%from the original authors if they were designed to select single stars
%with reliable rotation periods.  The final combination of these
%filters with our own flag for possible binarity yields our sample of
%benchmark rotators.






\section{Results}
\label{sec:results}

\subsection{Ages for stars in the Kepler field}
%TODO: show parameter distributions?

%e.g. hist_samples_field_gyro_ages_hist_field_gyro_ages_20230529.pdf

% compare against Berger+20, 
% note their sentence
% "Encouragingly, the distribution also qualita- tively matches the red giant asteroseismology-derived age distributions in Silva Aguirre et al. (2018) and Pinsonneault et al. (2018), as well as the rotation-based ages in Claytor et al. (2020) and the Galactic Archaeology with HERMES–Gaia ages in Buder et al. (2019)."
%
% also compare against Mor2019 and their literature assessment
% "Our findings that the thin disc SFH does not follow a sim- ple decreasing shape until the present are in good agreement with Snaith et al. (2015), and Haywood et al. (2016, 2018) who found, using data with metallicities and assuming a fixed IMF, the existence of an SFR quenching followed by a reactivation. Kroupa (2002a), using stellar kinematics, found the SFH to behave similarly. The relative maximum of the SFR that we find at 2–3 Gyr ago is compatible with the results of Vergely et al. (2002) and Cignoni et al. (2006) that, using Hipparcos data in a sphere of 80 pc around the Sun and assuming a fixed IMF, found maximum peaks at 1.75–2 Gyr ago and 2–3 Gyr ago, respec- tively. Recently, Bernard (2018), in a tentative work using TGAS data, pointed towards the existence of a relative maximum also located 2–3 Gyr ago."


\subsection{Ages for KOIs}

\subsection{Evolutionary trends in time}

{\it Kelvin-Helmholtz cooling}
$R_p \propto t^{-0.1}$ scaling \citep{Gupta_2019}

{\it Early carving of the photoevaporation desert}
\citep{Owen_Lai_2018}

{\it Rapidly decreasing abundance of big puffy planets}
Kepler not enough?

{\it Mini-Neptunes turning into Super-Earths}
\citep[e.g.][]{Rogers_2021}

{\it Boundary of the Super-Earth population moving up in time}
\citep{David_2021}


\section{Discussion}
\label{sec:disc}

\subsection{Isochrone Age Comparison}

\subsection{Lithium Age Comparison}

\subsection{Gyrochronal Age Comparison}
Searching the literature for gyrochronal analyses of the Kepler field,
the most relevant studies seemed to be those of
\citet{Walkowicz_2013}, \citet{Reinhold_2015}, and 
\citet{David_2021}.

\subsection{Asteroseismic Age Comparison}
T Ceillier, J Van Saders et al 2016 MNRAS...


\section{Conclusions}
\label{sec:conclusions}

\acknowledgements
This work was supported by the 
Heising-Simons 51~Pegasi~b Fellowship (LGB)
and the Arthur R.~Adams SURF Fellowship (EKP).

L.G.B.~conceived the project, collected HIRES spectra, executed the
rotation- and lithium-based age analyses, and drafted the manuscript.
E.K.P.~contributed to the rotation-based age analysis.
L.A.H.~contributed to project design.
H.I. and A.W.H~contributed to acquisition, reduction, and analysis of
the HIRES data.
All authors assisted in manuscript revision.


\facilities{
  Gaia \citep{Gaia_DR3_2022},
  Kepler \citep{Borucki10},
  TESS \citep{ricker_transiting_2015},
  NGTS \citep{Wheatley_2018}
}

\software{
    astropy \citep{Astropy18},
    matplotlib \citep{matplotlib},
    numpy \citep{numpy},
    scipy \citep{scipy},
}

%\clearpage

\bibliographystyle{aasjournal}
\bibliography{bibliography}

%\appendix
%\section{Can you avoid appendices?}
%\label{app:interp}

\clearpage
\listofchanges

\end{document}
