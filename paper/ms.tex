% upon AAS submission
%\documentclass[12pt,twocolumn,tighten,linenumbers]{aastex63}
%\documentclass[12pt,twocolumn,tighten,linenumbers,trackchanges]{aastex63}
% drafting / arxiv
\documentclass[11pt,twocolumn,tighten]{aastex63}
\turnoffedit

\usepackage{apjfonts}
\usepackage{url}
\usepackage{hyperref}
\usepackage{natbib}
\usepackage{amsmath,amstext,amssymb}
\usepackage[caption=false]{subfig} % for subfloat
\usepackage{xcolor, fontawesome}
\usepackage{color}
\usepackage{enumitem}

\newcommand{\red}{\color{red}}

\newcommand{\rprs}{{$R_p/R_{\star}$}}
\newcommand{\vsini}{{$V \sin i$}}
\newcommand{\kms}{{km\,s$^{-1}$}}
\newcommand{\gcc}{{g\,cm$^{-3}$}}
\newcommand{\rstar}{{$R_\star$}}
\newcommand{\rhostar}{{$\rho_\star$}}
\newcommand{\mearth}{{M$_\oplus$}}
\newcommand{\rearth}{{R$_\oplus$}}
\newcommand{\rsun}{{$R_\odot$}}
\newcommand{\msun}{{$M_\odot$}}
\newcommand{\bprp}{G_{\rm BP} - G_{\rm RP}}

\newcommand{\minus}{\scalebox{0.5}[1.0]{$-$}}

%%%%%%%%%%%%%%%%
% INSTITUTIONS %
%%%%%%%%%%%%%%%%
\newcommand{\caltech}{Department of Astronomy, MC 249-17, California Institute of Technology, Pasadena, CA 91125, USA}
\newcommand{\mitkavli}{MIT Kavli Institute and Department of Physics, 77 Massachusetts Avenue, Cambridge, MA 02139}
\newcommand{\berkeley}{Astronomy Department, University of California, Berkeley, CA 94720, USA}

%%%%%%%%%%%%%%%%%%%%%%%%%%%%%%%%%%%%%%%%%%
%%%%%%%%%%%%%%%%%%%%%%%%%%%%%%%%%%%%%%%%%%
%%%%%%%%%%%%%%%%%%%%%%%%%%%%%%%%%%%%%%%%%%

%
% ms specific numbers
%

%%%%%%%%%
% STARS %
%%%%%%%%%
\newcommand{\nuniqstarsantosrot}{54{,}148}
\newcommand{\nuniqstarsantosrotgyroappl}{26{,}614}
\newcommand{\nnonfpcumkois}{4{,}725}
\newcommand{\nnonfpcumkoihosts}{3{,}611}
\newcommand{\nplwgyroage}{910}
\newcommand{\nplhoststarwgyroage}{679}
\newcommand{\nplwgyroagenograzing}{697}
\newcommand{\nplhoststarwgyroagenograzing}{546}
\newcommand{\nuniqstarsantosrotteffcut}{45{,}658}
\newcommand{\ratiombtoybstars}{1.3}
\newcommand{\ratioobtoybstars}{2.1}
\newcommand{\ratiombtoybplanets}{1.7}
\newcommand{\ratioobtoybplanets}{2.5}
\newcommand{\nybstars}{3407.6}
\newcommand{\nmbstars}{4563.1}
\newcommand{\nobstars}{7084.3}
\newcommand{\nplyounggyro}{98}
\newcommand{\nplhostsyounggyro}{77}
\newcommand{\nplmidgyro}{167}
\newcommand{\nplhostsmidgyro}{123}
\newcommand{\nploldgyro}{242}
\newcommand{\nplhostsoldgyro}{187}
\newcommand{\nplyounggyrotwosigma}{54}
\newcommand{\nplhostsyounggyrotwosigma}{43}
\newcommand{\nplyounggyrothreesigma}{35}
\newcommand{\nplhostsyounggyrothreesigma}{31}


% number of stars monitored by Kepler (quoting Santos19/21)
\newcommand{\nkeplerstars}{$\approx$160{,}000}
% fraction of stars with rotation periods, with teff/logg in Berger+20
\newcommand{\fracstarswithprotwithbtwenty}{{$\sim$94\%}}
% complement of above
\newcommand{\fracstarswithprotwithoutbtwenty}{{$\sim$6\%}}

%%%%%%%%%%%
% PLANETS %
%%%%%%%%%%%
% number of KOIs in Q1-Q17 DR25 table
\newcommand{\nkois}{{$\sim$8{,}000}} %FIXME precise
% number of KOIs in Q1-Q17 DR25 table that are not known false positives
\newcommand{\nkoisnofp}{{$\sim$4{,}000}} %FIXME precise
% number of KOIs in Q1-Q17 DR25 table that have rotation periods
\newcommand{\nkoiswithprot}{{$\sim$2{,}000}}
% number of KOIs in Q1-Q17 DR25 table that are not known false positives that have rotation periods
\newcommand{\nkoisnofpwithprot}{{$\sim$1{,}000}}
% fraction of KOIs that are not FPs with B+20 parameters
\newcommand{\frackoisnofpwithprotwithbtwenty}{{$\sim$92\%}}

%%%%%%%%%%%
% LITHIUM %
%%%%%%%%%%%
% number of KOIs that are not known FPs that have a HIRES spectrum
\newcommand{\nkoiswithhires}{{1{,}500}}
% number of KOIs for which this work and Berger2018 overlap
\newcommand{\nbergeroverlap}{{500}}

%%%%%%%%
% AGES %
%%%%%%%%
\newcommand{\nkepstarswithages}{NNN}
\newcommand{\nkoisnofpwithages}{NNN}
\newcommand{\nkoisgoldwithages}{NNN}

%%%%%%%%%%%%%
% OFT-CITED %
%%%%%%%%%%%%%
\defcitealias{McQuillan_2014}{M14}
\defcitealias{Mazeh_2015}{M15}
\defcitealias{Santos_2019}{S19}
\defcitealias{Santos_2021}{S21}

%%%%%%%%%%%%%%%%%%%%%%%%%%%%%%%%%%%%%%%%%%
%%%%%%%%%%%%%%%%%%%%%%%%%%%%%%%%%%%%%%%%%%
%%%%%%%%%%%%%%%%%%%%%%%%%%%%%%%%%%%%%%%%%%

\begin{document}

%\title{Gyrochrone and Lithium Ages for Stars and Planets Observed by Kepler}
\title{Ages of Planets and Stars Observed by Kepler}

\correspondingauthor{Luke G. Bouma}
\email{luke@astro.caltech.edu}

\received{---}
\revised{---}
\accepted{---}
\shorttitle{gyro-johannes} 

\shortauthors{Bouma et al.}

\author[0000-0002-0514-5538]{Luke~G.~Bouma}
\altaffiliation{51 Pegasi b Fellow}
\affiliation{\caltech}

\author{Authors to include (order after LGB not assigned meaning;
currently alphabetical)}

% Early experiments for Kepler rotation samples
\author[0000-0001-7967-1795]{Elsa~K.~Palumbo}
\affiliation{\caltech}

% ?
%\author{Lynne~A.~Hillenbrand}
%\affiliation{\caltech}

% HIRES observations and reductions
\author[0000-0002-0531-1073]{H. Isaacson}
\affiliation{\berkeley}
% HIRES observations, reductions, JUMP infrastructure
\author[0000-0001-8638-0320]{A. W. Howard}
\affiliation{\caltech}
%
\author{HIRES observers to be named}
\affiliation{Various}

% <250 words
\begin{abstract}
  We report empirically calibrated rotation- and lithium-based ages
  for planets and stars observed by the primary Kepler mission.
  %
  Stellar rotation periods are collected from previous analyses of
  Kepler data; lithium content is assessed using both new and
  archival HIRES spectra.
  %
  We report ages for \nkoisgoldwithages\ FGK stars with transiting
  planets, with relative statistical precisions $\sigma_t / t$ of XX
  +YY -ZZ\%.  This is a factor of X.X +Y.Y -Z.Z more statistically precise than
  isochronal ages for the same stars, and $\approx$X.X times less
  precise than cluster-based ages.
  %
  The results include XXX planets younger than 1\,Gyr (at 3$\sigma$),
  and YYY planets younger than 0.5\,Gyr (at 3$\sigma$).
  %
  We also report rotation-based ages for XX,XXX stars observed
  by Kepler, aka.~the ``target star sample''.
  %
  In the latter sample of dwarf FGK stars observed by Kepler, we find
  that the gyrochrone age distribution rises linearly from 0 to
  3\,Gyr, with stars 2\,Gyr old being twice as common in the Kepler
  field as stars 1\,Gyr old.
  %
  This qualitatively agrees with previous analyses of the Milky Way's star
  formation history, though it offers greater sensitivity in the range of
  $\approx$0.1-1\,Gyr.
  %
%  The sample of detected Kepler planets has a similar age
%  distribution, though it is slightly deficient at $\lesssim$0.2\,Gyr,
%  likely due to decreased detection sensitivity.
  %
  Combining our stellar age catalog with Kepler's completeness products, we perform
  a hierarchical bayesian analysis to infer the average number of planets per star
  as a function of both stellar and planetary parameters.
  %
  While we recover many previously known correlations with stellar mass and metallicity,
  our analysis also highlights the importance of age in shaping the ``sub-Saturn desert'',
  and in influencing the evolution of the mini-Neptune and super-Earth populations.
  %TODO will this be the case?
%  We also comment on a few previously reported evolutionary trends in
%  the size evolution of close-in mini-Neptunes (2-4\,\rearth) and
%  super-Earths (1-2\,\rearth), a few of which we recover at comparable
%  ($\sim$3$\sigma$) significance to previous work, and one of which we
%  note for the first time.
\end{abstract}

\keywords{Exoplanet evolution (491), Stellar ages (1581)}

\section{Introduction}
\label{sec:intro}

Discovering a young planet requires solving two problems: find the
planet, and measure the star's age.  Each problem can be solved in
various ways, and so there are many paths for discovery.  In this
manuscript, we will focus on planets discovered using transits, and
stellar ages measured based using empirical rotation and lithium-based
methods.  Rotation-based age-dating will provide the most information
for the largest number of stars.  The lithium-based technique will
provide useful information for the subset of stars younger than a few
hundred million years.

To help build intuition, it is useful to assume that the age
distribution of stars within the nearest few hundred parsecs of the
Sun (i.e.~in the Solar Neighborhood) is uniform between 0 and 10\,Gyr
\citep{Nordstrom_2004}.  We will revisit this assumption later in the
manuscript.  While the youngest 0.1\% of stars ($\lesssim$10\,million
years; Myr) are perhaps the most constraining for studies of planet
formation, they are extremely challenging targets for close-in
exoplanet detection 
\citep[see e.g.][]{Damasso_2020,Bouma_2020_ptfo,Donati_2020}.  Studies of
the early orbital, structural, and atmospheric evolution of exoplanets
are perhaps most technically feasible for the $\approx$1\% of stars
with ages between 10 and 100\,Myr, or perhaps even the 10\% of stars
younger than 1\,Gyr.  However, since stellar ages are roughly
uniformly distributed in the solar neighborhood, there is a limit on
the sample size of young planets that will be detectable in these age
regimes, particularly relative to their older breathren.

%TODO mention the RV approach?  at least the Hyades etc work...
A wide range of approaches have, over the past ten years, begun to
establish the subfield of young, close-in planet discovery.
Pioneering discoveries with the prime Kepler mission (hereafter,
``Kepler'') focused on stars with known ages---stars in clusters---and
searched them for transiting planets \citep{Meibom_2013}.  Planets
discovered in this way are assumed to have the same age as the host
cluster.  The
resulting ages are precise at the $\approx$10\% level.  K2
and TESS have yielded larger numbers of young planets in clusters than
Kepler, because each of these missions has observed a larger fraction
of the sky, and therefore more stars with known ages
\citep[e.g.][]{Mann_K2_25_2016,Mann_2017,Curtis_2018,Livingston_2018,David_2019,Bouma_2020_toi837,Rizzuto_2020,Plavchan_2020,Newton_2021,Nardiello_2022,Tofflemire_2021,Zhou_2022,Zakhozhay_2022,Wood_2023}.

A limitation of the ``cluster-first'' approach is that only
$\approx$1\% of stars within the nearest few hundred parsecs can
currently be associated with their birth cluster
\citep[e.g.][]{Zari_2018,CantatGaudin_2020,Kounkel_2020,Kerr_2021}.
This means that there are many ``young'' (0.05-1\,Gyr) stars in the
field that are omitted by cluster-only searches.
Of order 40 transiting planets are currently known to be younger than
1\,Gyr, requiring an age precision $t/\sigma_t > 3$ (see
Figure~\ref{fig:rp_period_age}).  However, Kepler discovered
$\approx$4{,}000 planets; and so unless there is a major discovery
bias, we would expect a larger fraction of the Kepler
planets---perhaps 10\%, or 400 planets---to be younger than 1\,Gyr.

\begin{figure}[!t]
	\begin{center}
		\leavevmode
		\includegraphics[width=0.49\textwidth]{f1.pdf}
	\end{center}
	\vspace{-0.6cm}
	\caption{
		{\bf Radii, orbital periods, and ages of transiting exoplanets}.
    Planets younger than 1\,Gyr with ages more precise than a
    factor of three are emphasized. 
    ({\bf cite relevant papers?})
    Parameters are from the \citet{PSCompPars}.
		\label{fig:rp_period_age}
	}
\end{figure}

Have these young Kepler planets already been found?  If they have, the
most likely studies to have found them would be those by
\citet{Berger_2020b_rpage}, \citet{David_2021} or
\citet{Petigura_2022}.  The \citeauthor{Berger_2020b_rpage} and
\citeauthor{Petigura_2022} studies focused on
isochronal age-dating of individual stars, which is most precise 
for stars whose luminosities and temperature separate them from the
main-sequence.  The \citeauthor{David_2021} study leveraged both isochronal
ages from an earlier analogous study
\citep{Fulton_Petigura_2018_cks_vii}, and, in their core analysis, a
relative gyrochrone age-ranking approach that sorted stars based on
whether they rotated faster or slower than a gyrochrone age of
$\approx$2.6\,Gyr \citep{Meibom_2015,Curtis_2020}.

Theoretical predictions for planet evolution in the Kepler sample are
most important in the context of mini-Neptunes and super-Earths, since
these are the most abundant planets.  Pertinent predictions include:
{\it a)} that mini-Neptunes should cool and contract, roughly
following a $R_p \propto t^{-0.1}$ scaling \citep{Gupta_2019}; {\it
b)} that large and close-in mini-Neptunes at the edge of the
photoevaporation desert should shrink over the atmospheric loss
timescale, $t_{\rm loss}$ \citep{Owen_Lai_2018}; {\it c)} that the
abundance of 4-8\,R$_\oplus$ planets with masses of 5-30\,M$_\oplus$
should decrease over $t_{\rm loss}$---the upper-envelope of the
mini-Neptune size distribution should corresponding decrease,
e.g.~from $\approx$4.5\,R$_\oplus$ at $\approx$200\,Myr to
$\approx$3.5\,R$_\oplus$ at $\approx$700\,Myr
\citep[e.g.][]{Rogers_2021}; {\it d)} that the super-Earth to
mini-Neptune ratio, $\theta$, should increase over $t_{\rm loss}$
\citep[e.g.][]{Rogers_2021}, with the mini-Neptune abundance
decreasing by at least a factor of two or three, and the super-Earth
abundance increasing in tandem; {\it e)} that the
upper boundary of the super-Earth population should increase over the
same timescale as $\theta$ is increasing, since more massive
super-Earth cores should be able to retain their envelopes for longer
than less massive cores.

The details of the predictions are flexible, in that they depend on
the distributions of planetary core masses, initial atmospheric mass
fractions, and initial entropies of the atmospheres after disk
dispersal.  For instance, different sources of heating---from the star
\citep{Owen_Wu_2013,Lopez_Fortney_2014,Jin_2014}, planetary interior
\citep{Gupta_2019}, or even giant impacts
\citep{Biersteker_Schlichting_2019}---can produce very different
histories for the size evolution of planets at these early times
\citep[e.g.][]{Owen_2020}.  Different internal compositions also alter
the predicted size evolution \citep[e.g.][]{2024NatAs.tmp...33B}.  The
generic prediction that the ``upper envelope'' of the mini-Neptune
size distribution should shrink due to Kelvin-Helmholtz cooling is
hard to avoid; predictions however concerning the ``radius valley''
and movement of planets across it are dependent on the timescales and
prevalence of atmospheric mass loss.  In specific scenarios for
initial atmospheric accretion \citep{Lee_2022} and early atmospheric
evolution \citep{2024MNRAS.529.2716R} the radius gap might form
without needing to invoke post-disk dispersal mass loss at all.  In
such scenarios, there could be very little evolution of the planetary
size distribution between 1 and 4\,\rearth\ over the first gigayear.

Can stellar age measurements provide evidence for or against
these different models?  Neither {\it (a)} nor {\it (b)} has yet had a
convincing---or even suggestive--- detection reported (see
\citealt{Petigura_2022}).  While an apparent over-abundance of large
very young planets ({\it (c)}) has been noted by multiple authors
\citep[e.g.][]{2017AJ....153...64M,Bouma_2020}, its presence is almost certainly
exacerbated by a selection effect against sub-Neptune sized planet
detection at $<$100\,Myr \citep{Zhou_2021,Bouma_2022b}.  Recent studies have
nonetheless found that when accounting for completeness, there is
$\approx$3$\sigma$ evidence that hot mini-Neptunes are more prevalent in
$\approx$100-500\,Myr old K2 \citep{2023AJ....166..248C} and TESS
\citep{2024arXiv240303261V} stars than in the older Kepler sample
\citep[e.g.][]{2020AJ....159..248K}.
%TODO: more kepler citations...

Regarding the observed mini-Neptune to super-Earth ratio, {\it (d)} and {\it
(e)} were explored by \citet{Berger_2020b_rpage} and the twin
\citet{David_2021} and \citet{Sandoval_2021} studies.  Both
\citet{Berger_2020b_rpage} and \citet{Sandoval_2021} reported that $\theta$
increases with time, at a statistical significance of $\approx$2-3$\sigma$.
For \citet{Berger_2020b_rpage}, this conclusion was reached by comparing the
sizes of 85 planets with median isochronal ages younger than 1\,Gyr with a
property-matched sample of planets older than 1\,Gyr.  In
\citet{Sandoval_2021}, the conclusion was reached by computing the ratio while
performing a Monte Carlo re-sampling procedure from the isochronal ages and
planetary sizes computed by \citet{Fulton_Petigura_2018_cks_vii}.  Finally,
effect {\it (e)} was reported by \citet{David_2021}, who found that the average
location of the radius valley shifted from $\approx$1.8\,R$_\oplus$ to
$\approx$2.0\,R$_\oplus$ between 1.8 and 3.2\,Gyr.

The fact that stellar ages---and especially isochronal ages---are
correlated with stellar masses and metallicities is a source of stress
for many in this subfield.  \citet{Sandoval_2021} for instance noted a
strong trend in $\theta$ as a function of stellar metallicity in
addition to the trend they noted in age, and cautioned that this could
also explain their observations.  Independently, \citet{Petigura_2018}
and \citet{Petigura_2022} have reported trends between small-planet
occurrence and both stellar metallicity and mass; these parameters
seem, at first glance, to be more important predictors of bulk changes
in the planetary population than age.  Given these known correlations,
it can be useful to consider population models in which variables such
as metallicity, mass, and age are added in succession
\citep[e.g.][]{Thorngren_2021}.  An alternative, and perhaps more
rigorous approach, is to jointly infer the number of planets per star
as a function of both planetary and stellar parameters, the latter
including age \citep[e.g.][]{2023AJ....166..209M}.  Such analyses are
most effective when the measurement precision of these different
parameters is as high as possible.

Gyrochronology offers ages precise to $\lesssim$30\% for FGK stars
between 0.5-4\,Gyr, and can also provide useful age limits at
earlier times \citep{Bouma_2023}.  The idea of using a star's
spin-down as a clock is quite old
\citep{Skumanich_1972,Noyes_1984,Kawaler_1989,Barnes03,Mamajek_2008,Angus_2015},
and physics-based models for the spin-down itself can clarify many
aspects of how the stellar winds, internal structure, and magnetic
dynamo all evolve
\citep[e.g.][]{Matt_2015,Gallet_Bouvier_2015,Spada_2020}.  Key
for our analysis is the fact that we can empirically calibrate
rotation-based ages using measured sequences of rotation periods
in open clusters
(\citealt{Bouma_2023}, and references therein).  This approach
ultimately ties the rotation age scale to the cluster age scale, and
is limited in precision by the intrisic scatter in the $P_{\rm rot}$
sequences, and in accuracy by systematic uncertainties in the evolving
spin-down rates, and in the cluster ages themselves.
The oldest cluster for which this method has been calibrated is currently
M67 \citep[$\approx$4\,Gyr][]{2022ApJ...938..118D}.

Lithium-based age-dating includes two qualitatively distinct regimes:
depletion boundary ages derived for M-dwarfs in star clusters, and the
less precise decline-based ages for individual field FGK stars
\citep{Soderblom_2010}.  Here we focus on the latter approach, which
relies on the observation that the Li abundances of
partially-convective stars gradually decline as they age
\citep[e.g.][]{2005A&A...442..615S}.  The theoretical explanation for
this decline is debated
\citep[e.g.][]{1995ApJ...441..865C,2010ApJ...716.1269D,2019MNRAS.485.4052C}
Empirical understanding has however recently improved due to the work
by \citet{Jeffries_2023}, who built a model for how the equivalent
width (EW) of the \ion{Li}{1} 6708\,\AA\ doublet evolves as a function
of stellar effective temperature and age for a set of 6{,}200 stars in
52 open clusters.  Two-sided lithium ages are useful for Kepler (FGK)
stars between $\approx$0.03-0.5\,Gyr, with a strong dependence on
spectral type since the K-dwarfs lose their surface lithium much
faster than G-dwarfs.  The precision of the ages reported by the
\citet{Jeffries_2023} method in this regime are in the range of
0.3-0.5~dex.

The methods by which we select the stellar and planet samples are 
discussed in Section~\ref{sec:selection}.
We summarize the origin of stellar parameters other than the ages in
Section~\ref{sec:stellarprops}, 
and describe and validate our age-dating methods in
Sections~\ref{sec:rotage} and~\ref{sec:liage}.
The population-level trends for both the parent stellar sample and
the planet sample are discussed in Section~\ref{sec:results}.
A few conclusions are offered in Section~\ref{sec:conclusions}.


\section{Sample Selection}
\label{sec:selection}

This work focuses on stars surveyed by Kepler for which ages can be
inferred using either stellar rotation, lithium, or both.  These
``age-dateable stars'' are a relatively small subset of the
\nkeplerstars\ Kepler targets.  Rotation periods provide the majority
of the information, since they have been reported for
$\approx$55{,}000 Kepler stars
\citep[e.g.][]{McQuillan_2014,Santos_2021}.  
High-resolution measurements of the \ion{Li}{1} 6708\,\AA\ doublet are
available for only a few thousand Kepler objects of interest (KOIs).
We therefore define our selection function as Kepler stars
with measured rotation periods ($\mathcal{S}$;
Section~\ref{subsec:starsel}), and Kepler planets whose host stars
have measured rotation periods ($\mathcal{P}$;
Section~\ref{subsec:planetsel}).


\subsection{Rotation periods}
\label{subsec:starsel}

We select stars with rotation periods that have been measured by
previous investigators.  Studies of stellar rotation across the entire
Kepler field include those by \citet{McQuillan_2014}
(\citetalias{McQuillan_2014}), \citet{Reinhold_2015},
\citet{Santos_2019} (\citetalias{Santos_2019}), and
\citet{Santos_2021} (\citetalias{Santos_2021}).
\citetalias{McQuillan_2014} used an approach based on the
autocorrelation function to detect 34{,}030 rotation periods for
main-sequence Kepler targets cooler than 6500\,K, and excluded known
eclipsing binaries and KOIs.  \citet{Reinhold_2015} used an iterative
Lomb-Scargle based approach to analyze $\approx$40{,}000 stars with a
variability range $R_{\rm var}>0.3$\% that were not known EBs, planet
candidates, or pulsators; they reported primary rotation periods for
24{,}124 of these stars.  \citet{Reinhold_2015} also reported
secondary periods for two thirds of the stars with primary periods,
which could be caused by differential rotation or finite spot
lifetimes.  Finally, \citetalias{Santos_2019} and
\citetalias{Santos_2021} combined a wavelet analysis and
autocorrelation-based approach, and cumulatively reported rotation
periods for 55{,}232 main-sequence and subgiant FGKM stars.
\citetalias{Santos_2019} and \citetalias{Santos_2021} included known
KOIs and binaries, and assigned them specific quality flags.  The
rotation periods of KOIs have received considerable additional
scrutiny
\citep[e.g.][]{Walkowicz_2013,Mazeh_2015,Angus_2018,David_2021}.

We choose to adopt the results of \citetalias{Santos_2019} and
\citetalias{Santos_2021} as our default rotation periods.
\citetalias{Santos_2021} provides a detailed comparison against
\citetalias{McQuillan_2014}; the brief summary is that the periods
agree for 99.0\% of the 31{,}038 period detections in common between
the two studies.  \citetalias{Santos_2021} classified the 2{,}992
remaining stars from \citetalias{McQuillan_2014} as not showing
rotation periods based on updated knowledge of contaminants
(e.g.~giant stars and eclipsing binaries) and visual inspection.
In addition,
\citetalias{Santos_2021} report rotation periods for 24{,}182
main-sequence and subgiant FGKM stars that were not reported as
periodic by \citetalias{McQuillan_2014}.  Many of
these reported detections have lower variability amplitudes, and
longer periods, than those reported by
\citetalias{McQuillan_2014}.  A relevant flag for stars
missing \citetalias{McQuillan_2014} periods is included in
Table~\ref{tab:stars}.  Reviewing the resulting period detections, we
learned that 32 KOIs with orbital and rotation periods within
$\approx$20\% were excluded from the \citetalias{Santos_2019} and
\citetalias{Santos_2021} catalogs (A.~Santos, priv.~comm.) ; we choose
to include these objects.\footnote{In general, excluding stars in this
category would impose a bias against young close-in planets, due to
the natural commensurability between the rotation rates of young
stars, and Kepler's increased sensitivity to short-period objects.
While some of these KOIs are known false positives, our analysis
naturally excludes such objects at later steps.  As one motivating
example, this subset of stars happens to include {\it
e.g.}~Kepler~1643 (KIC~8653134), with $P_{\rm orb}$=5.34\,days and
$P_{\rm rot}$=5.05\,days, which is known to be $\approx$40\,Myr old
based on membership in RSG-5 \citep{Bouma_2022b}.  }
The final tally of stars from \citetalias{Santos_2019},
\citetalias{Santos_2021}, and our corrected KOI list is
\nuniqstarsantosrot\ Kepler stars with rotation periods.

%  %TODO is it?
%  We verified that the main conclusions of our analysis do not change if
%  we were to adopt only the periods from \citetalias{McQuillan_2014}.
%  %TODO have you?




\subsection{Kepler objects of interest}
\label{subsec:planetsel}

We considered two different samples of planets: those in the fully
automated Q1-Q17 DR25 KOI Table \citep{Thompson_2018}, which is
suitable for planet occurrence rate calculations, and also those in
the cumulative KOI table, which included the best knowledge available
on any given planet candidate as of 2023 Jun 6, but incorporated
human-based vetting \footnote{See
\url{https://exoplanetarchive.ipac.caltech.edu/docs/PurposeOfKOITable.html}}.
For the main body of this work, we will refer to the former
homogeneous sample, which includes \nkois\ KOIs, \nkoisnofp\ that are
not known false positives; Appendix~\ref{app:cumulativekoi} highlights
some additional young planets that are omitted in this approach.

Crossmatching the Q1-Q17 DR25 KOI table against our stars with
rotation periods yields \nkoiswithprot\ KOIs with rotation periods, of which \nkoisnofpwithprot\ 
are not known false postives.
These objects will comprise the core data for our gyrochronology analysis.

As a sanity check on the statistical uncertainties of these rotation
periods, we compared our adopted \citetalias{Santos_2019} and
\citetalias{Santos_2021} periods with those reported by
\citet{McQuillan_2014} and \citet{Mazeh_2015}.
We found that for $P_{\rm rot}\lesssim$15\,days, the two 
datasets agree at a precision of $\lesssim$0.01$P_{\rm rot}$.  At longer periods of 
$P_{\rm rot}\approx$30\,days, the agreement was typically at the
$\lesssim$0.03$P_{\rm rot}$ level, and
the upper envelope of the period difference tended to increase
linearly with period.
Based on this comparison, we adopted a simple prescription for the
period uncertainties,
such that there are 1\% relative uncertainties below $P_{\rm
rot}=15$\,days, and a linear increase thereafter, with slope set to
guarantee 3\% $P_{\rm rot}$ uncertainties at 30 days.



\subsection{Lithium sample}
A final subcomponent of our analysis involves an age assessment based on
the equivalent width of the \ion{Li}{1} 6708\,\AA\ doublet.  For this
component of the work, we primarily rely on spectra acquired with the
High Resolution Echelle Spectrometer (HIRES;
\citealt{vogt_hires_1994}) on the Keck I 10m telescope.
Most of these spectra were acquired between 2010 and 2020
as part of the California Keck Survey (CKS).
A subset were acquired by our team in our own observing programs.
%TODO quantify.
%TODO compare to berger2018?

These HIRES spectra are available for \nkoiswithhires\ of the
\nkoisnofp\ non-false-positive KOIs.
In the broader stellar sample, they are available for a much smaller 
fraction of the stars observed by Kepler, because they were generally only
acquired due to the presence of a transiting planet candidate.
Given this selection function,
we leverage these spectra in our analysis as a way to scrutinize the
ages of just the planet sample.



\section{Stellar Properties}
\label{sec:stellarprops}

% https://gea.esac.esa.int/archive/documentation/GDR3/Data_analysis/chap_cu8par/sec_cu8par_apsis/ssec_cu8par_apsis_gspphot.html

As our default source for stellar temperatures and surface gravities,
we adopt the values from the Gaia-Kepler Stellar Properties Catalog
(GKSP; \citealt{Berger_2020a_catalog}).  The GKSP parameters are
available for stars with ``AAA'' 2MASS photometry, measured parallaxes
in Gaia DR2,  and $g$-band photometry from either the KIC or the
Kepler-INT survey.  The parameters themselves were derived using
\texttt{isoclassify} \citep{2017ApJ...844..102H} to interpolate over
the MIST isochrone grids
\citep{2016ApJ...823..102C,2016ApJS..222....8D}, given the SDSS $g$
and 2MASS $K_{\rm s}$ photometry, the Gaia DR2 parallaxes, and
$V$-band extinction from the \citet{2018MNRAS.478..651G} reddening
map.  The resulting stellar parameters are available for
\fracstarswithprotwithbtwenty\ of the \nuniqstarsantosrot\ Kepler
stars with rotation periods.

%For the remaining \fracstarswithprotwithoutbtwenty\ of stars that lack
%temperatures and surface gravities, we adopt the values 
%from the Gaia DR3 General Stellar Parametrizer from Photometry (GSP-Phot; CITE).
%GSP-Phot uses the low-resolution Gaia BP and RP spectra, mean $G$ magnitudes,
%and parallaxes, in concert with the PARSEC isochrones (CITE Tang2014, Chen2015,
%Pastorelli2020) to infer the stellar effective temperatures, surface gravities,
%metallicities, and line-of-sight extinctions.
%In the planet sample,
%\frackoisnofpwithprotwithbtwenty\ of the
%the non-false-positive KOIs with rotation periods have parameters from
%\citet{Berger_2020a_catalog}.

For the remaining \fracstarswithprotwithoutbtwenty\ of stars that lack
temperatures and surface gravities, we adopt the values reported by
\citet{Santos_2019} and \citet{Santos_2021}, which for these cases
primarily derive from the \citet{Mathur_2017} DR25 Kepler Stellar
Properties Catalog, and are mostly derived from photometry.  In the
planet sample, \frackoisnofpwithprotwithbtwenty\ of the the
non-false-positive KOIs with rotation periods have parameters from
\citet{Berger_2020a_catalog}, and the remainder similarly are drawn
from DR25. 

As a sanity check on our effective temperature scale, we then used the 
the CITE Green2019 reddening map, combined with the Gaia DR2 $G_{\rm
BP}-G_{\rm RP}$ color, to calculate effective temperatures using the
relation suggested by \citet{Curtis_2020}.
This latter effective temperature scale was demonstrated to have a 
scatter of $\approx$50\,K for stars near the main-sequence.
We find {\bf X, Y, Z.}
We ultimately favor the \citet{Berger_2020a_catalog} scale
because it uses 2MASS $K_{\rm s}$ photometry, and is tied to an actual
isochronal fit.

%TODO: plot uncertainties in teff/logg for both parameter sets.

% TODO: add that plot showing differences btwn the two teff scales?


\section{Ages From Rotation}
\label{sec:rotage}

We calculate gyrochrone ages using \texttt{gyro-interp}
\citep{Bouma_2023}.  We built this statistical framework to address
the issue that, especially over the first gigayear of stellar
spin-down, stars with the same mass and same rotation period can have
very different ages \citep[e.g.][]{Curtis_2019_ngc6811}.  The
gyrochrone age posterior should therefore incorporate this intrinsic
population-level scatter into the precision with which the age can be
measured.  The three main assumptions behind the \texttt{gyro-interp}
framework are as follows.  {\it 1)} The stars are on the main
sequence, {\it 2)} their spindown is not influenced by binary
companions, and {\it 3)} the rotation period distribution for field
stars of a given mass and age is identical to that of the calibration
open clusters (metallicity differences, for instance, are ignored).  
Regarding the former two assumptions: pre-main-sequence and
post-main-sequence evolution introduce variations in the stellar
angular momentum budget.  Similarly, tidal effects might
systematically spin up (or down) stars in binaries.

Using our adopted effective temperatures and rotation periods, we ran
\texttt{gyro-interp} over an age grid of 0 to 4\,Gyr, linearly spaced
over 500 grid points.  \texttt{gyro-interp} itself is only calibrated
out to 4\,Gyr, the adopted age of the M67 cluster
\citep[see][]{Dungee_2022,Gruner_2023}.  To estimate the stellar
spin-down rate as a function of time and stellar temperature, this
framework interpolates between reference open clusters using smooth
cubic hermite polynomials.  We calculate the probability of the
rotation-based age $t_{\rm gyro}$, given the observed rotation
periods, effective temperatures, and their uncertainties by
integrating Equation~1 of \citep{Bouma_2023} at the default grid
resolution.  This procedure marginalizes over the intrinsic age and
temperature dependent scatter that is observed in the cluster
sequences to yield statistical age uncertainties.


\subsection{Star Quality Flags}
\label{subsec:flags}
We calculated gyrochrone age posteriors for all \nuniqstarsantosrot\
stars with reported rotation periods and effective temperatures.  To
select the subset stars for which we expect these age to in fact be
valid, we then build a set of quality flags.  When and
how these flags should be applied depends on the question being asked.
If the goal is to construct a false-positive free sample, they might
all be applied.
However if the goal is constructing a complete sample of young stars,
then consider the examples of Kepler~1627Ab ($\approx$40\,Myr) and
Kepler~51c ($\approx$625\,Myr).  The former has a high RUWE due to a
resolved binary companion \citep{Bouma_2022a}; the latter is on a
grazing orbit \citep{2014ApJ...783...53M}.  
The flags in Tables~\ref{tab:stars} and~\ref{tab:planets} enable
selecting for such cases using bitflags.

{\it Subgiants}---We flagged stars as possible subgiants if they had
$\log g < 4.2$.  We set this threshold by examining the
relevant flag from \citet{berger_2018_radii_evolnstates},
cross-matched against \citet{Berger_2020a_catalog}.
\citet{berger_2018_radii_evolnstates} performed this classification
using the PARSEC evolutionary models, stellar radii, and DR25 stellar
properties catalog effective temperatures \citep{Mathur_2017}.

{\it Photometric binaries}---We plotted color--absolute magnitude
diagrams using the Gaia DR2 photometry, in $M_{\rm G}$ vs. $(G_{\rm
BP} - G_{\rm RP})_0$, and $(G - G_{\rm RP})_0$.  We then manually drew
loci to flag over or under-luminous stars in each diagram, typically
more than $\approx$0.5 magnitudes above or below the main sequence.
%TODO refer to a figure!!!

{\it RUWE}---Following recommendations in the Gaia DR3
documentation\footnote{\url{https://gea.esac.esa.int/archive/documentation/GDR2/Gaia_archive/chap_datamodel/sec_dm_main_tables/ssec_dm_ruwe.html}},
we flagged stars with a Gaia DR3 renormalized unit weight error (RUWE)
exceeding 1.4 as possible binaries.  Such astrometric outliers can be
either bonafide astrometric binaries, or more often are marginally
resolved point sources for which the single-source PSF model used in
the Gaia pipeline induces spurious astrometric scatter.

{\it Crowding}---We searched the Gaia DR3 point source catalog for
stars within 1 Kepler pixel (4$''$) of every target star.
While many such companions are not physically associated with the
target star, their presence can bias stellar parameters and stellar
rotation period measurements.
We therefore flag any stars with neighbors down to 10$\times$ the 
brightness of the target star within this region ($\Delta G < 2.5$).

{\it Gaia DR3 Non-Single-Stars}---The Gaia DR3
\texttt{non\_single\_star} column in the \texttt{gaia\_source} table
flags known eclipsing, astrometric, and spectroscopic binaries.  We
directly include this column.

{\it Final calibration sample}---{\bf todo: mention any extra flags
from Berger2020, or perhaps in the KOI catalog itself, or perhaps the
amplitude cuts you might want to apply to the rotation period catalog}


\subsection{Planet Quality Flags}
For the subset of our analysis that focuses on the properties of
planets, we use the following additional quality flags.

% see Erik's 2022 discussion; might want to shift to include a cut on
% disposition score.
{\it Candidate reliability}---no false positives; only ``planet
candidates'' and ``confirmed planets''.

{\it Candidate S/N}---the false positive rate increases greatly toward
Kepler's noise floor for planet detection (CITE).  We required a S/N
in excess of the usual floor, through a cut on the ``multiple event
statistic'' (MES): we required ${\rm MES}$$>$10.

{\it Grazing planets}---grazing objects ($b$>0.8) often provide
non-robust planetary sizes due to the size-impact parameter
degeneracy.  Particularly for large planets, they can also include
astrophysical false positives at higher rates (due to the same
degeneracy).  While we include a flag for such objects, we by default
do not exclude them from our analysis; this is because...{\bf justify}.

\subsection{Tally}
\label{subsec:tally}

\begin{figure*}[!t]
	\begin{center}
		\subfloat{
			\includegraphics[width=0.5\textwidth]{prot_teff_Santos19_Santos21_clean0.pdf}
		}
		\subfloat{
			\includegraphics[width=0.5\textwidth]{koi_mean_prot_teff_koi_X_S19S21dquality_keepgrazing.pdf}
		}
	\end{center}
	\vspace{-0.5cm}
	\caption{
    {\bf Rotation periods for Kepler stars (left) and planets (right)}.
    The periods are from \citet{Santos_2019} and \citet{Santos_2021}
    (\citetalias{Santos_2019} and \citetalias{Santos_2021}), and the
    temperatures are primarily from \citet{Berger_2020a_catalog}.  The
    gray lines are ``mean fits'' from \citet{Bouma_2023} to the slow
    rotation sequences of open clusters.  The stellar sample shows
    only apparently single stars near the main sequence with $\log
    g$$>$4.2, RUWE$<$1.4, 3800$<$$T_{\rm eff}$/K$<$6200, that are not
    flagged as eclipsing binaries.  The planet sample has the same
    stellar cuts, and shows confirmed and candidate planets from the
    cumulative KOI table, requiring them to have a MES$>$10.
	}
	\label{fig:prot_vs_teff}
\end{figure*}

The rotation periods for the stellar and planetary samples are shown
in Figure~\ref{fig:prot_vs_teff}.  Out of the \nkeplerstars\ stars
observed by Kepler, requiring reported rotation periods yielded
\nuniqstarsantosrot.  Further requiring these rotating stars
to have temperatures 3800$<$$T_{\rm eff}$/K$<$6200, which is the range
over which \texttt{gyro-interp} returns finite age posteriors,
yields \nuniqstarsantosrotteffcut\ stars.
Further requiring them to be apparently single, near the main
sequence, with $\log g$$>$4.2, and not flagged as
eclipsing binaries yielded \nuniqstarsantosallbutruwe\ rotators
potentially amenable for gyrochronology.
Finally, imposing RUWE$<$1.4 leaves \nuniqstarsantosrotgyroappl\ stars
for which gyrochronology is likely to be valid.


In the planet sample, the cumulative KOI table included
\nnonfpcumkois\ confirmed and candidate planets, orbiting
\nnonfpcumkoihosts\ stars.  Assume that we apply all the same stellar
cuts, save for that on RUWE.
Then,
require these confirmed and candidate planets to have MES$>$10.
This combination of selection criteria
yield \nplwgyroagewithgrazingandhighruwe\ planets with rotation-based
ages orbiting \nplhoststarwgyroagewithgrazingandhighruwe\ stars
(\nplhoststarwgyroagejusthighruwe\ of these stars with ``high'' RUWE).
If we additionally require the planets to be non-grazing ($b<0.8$)
based on the Markov Chain Monte Carlo (MCMC) fits from
\citet{Thompson_2018}, we are left with \nplwgyroagenograzing\ and
\nplhoststarwgyroagenograzing\ planets and stars, respectively.




\section{Ages From Lithium}
\label{sec:liage}

\begin{figure}[!t]
	\begin{center}
		\leavevmode
		\includegraphics[width=0.49\textwidth]{li_vs_teff_eagles_showdispersion.pdf}
	\end{center}
	\vspace{-0.6cm}
	\caption{
    {\bf Equivalent widths (EW) of the \ion{Li}{1} 6708\,\AA\ doublet
    from Keck/HIRES spectra collected between 2009 and 2024.}
    Lines represent mean models derived from Gaia-ESO spectra of stars
    in clusters of various ages; the intrinsic dispersion 
    evolves in time, and is modelled as the $\pm$1$\sigma$
    bands~\citep{Jeffries_2023}.
    As for rotation-based ages, the intrisic dispersion sets the
    empirical precision limit for the technique.
		\label{fig:li_vs_teff}
	}
\end{figure}

%TODO: does JUMP mirror all HIRES data from KOA?
We queried a combination of the Keck Observatory Archive and the
internal CPS archive to collect all HIRES spectra that were available
for the \nplhoststarwgyroagewithgrazingandhighruwe\ stars
described in Section~\ref{subsec:tally} for which it might be possible
to derive rotation-based ages.

%
We required the spectra to not have the iodine cell in...
%TODO ?!?!?!
%

This exercise yielded at least one spectrum for \nlithiumstars\ and 
\nlithiumplanets\ planets.
These observations were acquired between 6 September 2009 and 18
September 2022.


Out of the \nplhoststarwgyroagewithgrazingandhighruwe\ stars for w

To measure the lithium equivalent widths for the \nkoiswithhires\ KOIs
with HIRES spectra, we use the procedure described by
\citet{Bouma_2021}.
The stars of interest for this work are FGK stars, and so the
continuum in the vicinity of the
\ion{Li}{1} 6708\,\AA\ doublet is well-defined.
We therefore numerically integrate a single best-fit Gaussian over a
local window, and estimate the uncertainties on the line width through
a Monte Carlo procedure that bootstraps against the local scatter in the
continuum.
Generally speaking, this approach does not correct for the neighboring
\ion{Fe}{1} 6707.44\,\AA\ blend.

To evaluate the accuracy and precision of this approach, after
applying an initial iteration of this method on the \nkoiswithhires\
KOIs with HIRES spectra, we compared our lithium equivalent widths
with those reported by \citet{Berger_2018}.
For the \nbergeroverlap\ stars in both samples, 
we found {\bf X, Y, Z.}

To calculate lithium ages, we then use EAGLES \citep{Jeffries_2023}.
{\bf TODO: explain why}
%TODO: explain why.







\section{Results}
\label{sec:results}


\subsection{Age distribution}

\begin{figure*}[!t]
	\begin{center}
		\leavevmode
		\includegraphics[width=0.9\textwidth]{hist_samples_koi_gyro_ages_hist_field_gyro_ages_20240405_maxage3200.pdf}
	\end{center}
	\vspace{-0.6cm}
	\caption{
		{\bf Age distribution of stars and planet-hosts in the Kepler field
    inferred from rotation periods}.
    A histogram of samples from the age posteriors for all Kepler
    stars with reported rotation periods and temperatures is shown in
    light gray;
    the age calculation was actually valid for stars that meet the criteria
    discussed in Section~\ref{subsec:flags} (dark blue).
		\label{fig:hist_tgyro}
	}
\end{figure*}

Figure~\ref{fig:hist_tgyro} shows the gyrochone age distributions of
stars and planet-hosts in the Kepler field.
We truncated the plots near $3$\,Gyr due to arguments that suggest the
detectability of rotation signals is near-unity up to this age for
Sun-like stars, and decreases for older ages
\citep{2022ApJ...937...94M}.
Both the stellar and planetary samples point toward a star formation
rate in the Kepler field that has slowed over the past few gigayears.
{\it LGB pending words: allude to findings on MW SFH from chemistry,
isochrones, and asteroseismology}.

% compare against Berger+20, 
% note their sentence
% "Encouragingly, the distribution also qualita- tively matches the red giant asteroseismology-derived age distributions in Silva Aguirre et al. (2018) and Pinsonneault et al. (2018), as well as the rotation-based ages in Claytor et al. (2020) and the Galactic Archaeology with HERMES–Gaia ages in Buder et al. (2019)."
%
% also compare against Mor2019 and their literature assessment
% "Our findings that the thin disc SFH does not follow a sim- ple decreasing shape until the present are in good agreement with Snaith et al. (2015), and Haywood et al. (2016, 2018) who found, using data with metallicities and assuming a fixed IMF, the existence of an SFR quenching followed by a reactivation. Kroupa (2002a), using stellar kinematics, found the SFH to behave similarly. The relative maximum of the SFR that we find at 2–3 Gyr ago is compatible with the results of Vergely et al. (2002) and Cignoni et al. (2006) that, using Hipparcos data in a sphere of 80 pc around the Sun and assuming a fixed IMF, found maximum peaks at 1.75–2 Gyr ago and 2–3 Gyr ago, respec- tively. Recently, Bernard (2018), in a tentative work using TGAS data, pointed towards the existence of a relative maximum also located 2–3 Gyr ago."

Imagine we were to group stars between 0-1\,Gyr, 1-2\,Gyr, and
2-3\,Gyr, and label them ``young'', ``intermediate-age'', and ``old''.
A simple counting exercise from the distributions in
Figure~\ref{fig:hist_tgyro} tells us that there are \ratioobtoybstars\
times as many old stars in the Kepler field as young stars.
Similarly, there are \ratioobtoybplanets\ times as many old planet
hosts as young planet hosts. 

In terms of detected planet counts, our rotation-based ages for
the Kepler sample yield \{\nplyounggyro, \nplmidgyro, \nploldgyro\}
detected planets in the 0-1\,Gyr, 1-2\,Gyr, and 2-3\,Gyr bins.
For the sub-gigayear bin in particular, which holds specific weight in
studies of planet evolution, while there are \nplyounggyro\ planets
with median ages below 1\,Gyr, requiring $t_{\rm
gyro}$$<$1\,Gyr at 2$\sigma$ confidence yields
\nplyounggyrotwosigma\ planets orbiting \nplhostsyounggyrotwosigma\
stars.

\subsection{Young Planets}




To date, four stellar clusters in the Kepler field have yielded planet
detections.
They are:...
Young Kepler systems with the most precisely measured ages are...

(TODO: perhaps easiest in a little table.  Theia316, Theia520, CepHer,
and NGC6811).




\subsection{Ages for KOIs}

\subsection{Evolutionary trends in time}

{\it Kelvin-Helmholtz cooling}
$R_p \propto t^{-0.1}$ scaling \citep{Gupta_2019}

{\it Early carving of the photoevaporation desert}
\citep{Owen_Lai_2018}

{\it Rapidly decreasing abundance of big puffy planets}
Kepler not enough?

{\it Mini-Neptunes turning into Super-Earths}
\citep[e.g.][]{Rogers_2021}

{\it Boundary of the Super-Earth population moving up in time}
\citep{David_2021}


\section{Discussion}
\label{sec:disc}

\subsection{``Galactic Archaeology'' Comparison}
Zasowski et al fitted a chemical-isochrone model to APOGEE, X, Y, Z
data and presented a model for the star formation history of the MW.
The gist is a spike at z=2, and a gradual decrease since then, by
about a few of ten.

Mor2019 did isochrone fitting to Gaia, and found the same, but with a
spike at 3Gyr.

Our inferred gyro-age distribution of the Kepler field seems to agree
with these pictures.  The main value added is likely over the first
gigayear.

\subsection{Isochrone Age Comparison}

\subsection{Lithium Age Comparison}

\subsection{Gyrochronal Age Comparison}
Searching the literature for gyrochronal analyses of the Kepler field,
the most relevant studies seemed to be those of
\citet{Walkowicz_2013}, \citet{Reinhold_2015}, and 
\citet{David_2021}.

Also Mathur2023, which does ``magneto-gyrochronology'', including the
Sph indicator in a model of the ages.


\subsection{Asteroseismic Age Comparison}
T Ceillier, J Van Saders et al 2016 MNRAS...


\section{Conclusions}
\label{sec:conclusions}

\acknowledgements
This work was supported by the 
Heising-Simons 51~Pegasi~b Fellowship (LGB)
and the Arthur R.~Adams SURF Fellowship (EKP).

L.G.B.~conceived the project, collected HIRES spectra, executed the
rotation- and lithium-based age analyses, and drafted the manuscript.
E.K.P.~contributed to the rotation-based age analysis.
L.A.H.~contributed to project design.
H.I. and A.W.H~contributed to acquisition, reduction, and analysis of
the HIRES data.
All authors assisted in manuscript revision.


\facilities{
  Gaia \citep{Gaia_DR3_2022},
  Kepler \citep{Borucki10},
  TESS \citep{ricker_transiting_2015},
  NGTS \citep{Wheatley_2018}
}

\software{
    astropy \citep{Astropy18},
    matplotlib \citep{matplotlib},
    numpy \citep{numpy},
    scipy \citep{scipy},
}

\clearpage 

\startlongtable
\begin{deluxetable}{lllrrrrr}
	%\tabletypesize{\footnotesize}
	\tabletypesize{\scriptsize}
	\tablecaption{Young planets and planet candidates.
		%The non-truncated machine readable versions are accessible both
		%through the online journal, and at
		%\url{https://zenodo.org/record/8327508}.
		\label{tab:youngplanets}}
	%\toprule
	%\midrule
	%\endhead
	\startdata
	KOI & Kepler & $t_{\rm gyro}$ & $R_{\rm p}$ & $P_{\rm orb}$ & $f_{\rm RUWE}$ & $f_{\rm grazing}$ & Spec? \\
  -- &   -- & Myr &    Earths &    days &      bool &   bool  & bool \\
	\hline
	K05245.01 & Kepler-1627 b & 5357 & 2.62 & $225\pm7$ & $80^{+152}_{-56} $ & $51^{+38}_{-27}$ & Yes & 3.79 & 7.2 & 0 & 1152 & 1 & Cep-Her \\
K07368.01 & Kepler-1974 b & 5068 & 2.56 & $248\pm4$ & $88^{+176}_{-64} $ & $54^{+47}_{-25}$ & Yes & 2.22 & 6.84 & 0 & 1536 & 1 & Cep-Her \\
K06228.01 & Kepler-1644 b & 5521 & 1.43 & $-2\pm13$ & $72^{+144}_{-48} $ & $> 767$ & No & 1.88 & 21.09 & 4 & 1666 & 1 & Unres. Binary \\
K06186.01 & Kepler-1643 b & 4918 & 5.05 & $120\pm6$ & $80^{+176}_{-56} $ & $191^{+92}_{-76}$ & Yes & 2.11 & 5.34 & 0 & 2048 & 1 & Cep-Her \\
K03933.01 & Kepler-1699 b & 5496 & 4.16 & $-11\pm7$ & $80^{+104}_{-56} $ & $> 889$ & No & 1.32 & 3.49 & 0 & 1152 & 1 & Unres. Binary \\
K03916.01 & Kepler-1529 b & 4974 & 6.43 & $200\pm6$ & $104^{+112}_{-72} $ & $90^{+53}_{-39}$ & Yes & 2.01 & 5.34 & 0 & 2048 & 1 & \checkmark \checkmark \\
K07913.01 & Kepler-1975 b & 4450 & 3.36 & $56\pm9$ & $96^{+216}_{-72} $ & $> 57$ & Yes & 2.03 & 24.28 & 0 & 1792 & 1 & Cep-Her \\
K01804.01 & Kepler-957 b & 4947 & 4.52 & $24\pm9$ & $96^{+192}_{-72} $ & $> 241$ & Yes & 6.9 & 5.91 & 0 & 0 & 1 & \checkmark \\
K03936.02 & Kepler-1930 b & 4906 & 7.1 & $170\pm4$ & $176^{+104}_{-64} $ & $115^{+55}_{-49}$ & Yes & 1.52 & 13.03 & 4 & 0 & 1 &  \\
K03876.01 & Kepler-1928 b & 5577 & 4.64 & $137\pm4$ & $144^{+104}_{-88} $ & $189^{+150}_{-94}$ & Yes & 1.86 & 19.58 & 0 & 1024 & 1 & MELANGE-3 \\
K04069.01 & Kepler-1938 b & 4617 & 7.82 & $6\pm16$ & $152^{+112}_{-40} $ & $> 208$ & Yes & 1.47 & 13.06 & 0 & 1024 & 1 & \checkmark \\
K02678.01 & Kepler-1313 b & 5236 & 6.13 & $142\pm3$ & $192^{+112}_{-88} $ & $174^{+96}_{-72}$ & Yes & 1.71 & 3.83 & 4 & 1024 & 1 &  \\
K04194.01 & Kepler-1565 b & 4958 & 7.4 & $133\pm7$ & $232^{+104}_{-88} $ & $174^{+81}_{-72}$ & Yes & 1.27 & 1.54 & 0 & 0 & 1 & \checkmark \checkmark \\
K03835.01 & Kepler-1521 b & 4806 & 7.82 & $117\pm5$ & $208^{+96}_{-72} $ & $176^{+79}_{-69}$ & Yes & 2.3 & 47.15 & 0 & 1024 & 1 & \checkmark \checkmark \\
K01838.01 & Kepler-970 b & 4314 & 9.23 & $36\pm14$ & $176^{+120}_{-40} $ & $> 92$ & Yes & 2.15 & 16.74 & 4 & 0 & 1 & MELANGE-3 \\
K00063.01 & Kepler-63 b & 5486 & 5.49 & $89\pm4$ & $224^{+96}_{-96} $ & $542^{+475}_{-256}$ & Yes & 5.64 & 9.43 & 0 & 2048 & 1 & \checkmark \checkmark \\
K01199.01 & Kepler-786 b & 4680 & 33.06 & $83\pm6$ & $3872^{+96}_{-168} $ & $228^{+168}_{-87}$ & No & 2.31 & 53.53 & 0 & 0 & 1 & Mystery \\
K03316.01 & Kepler-1467 b & 5252 & 6.31 & $122\pm6$ & $232^{+112}_{-104} $ & $236^{+151}_{-95}$ & Yes & 3.11 & 47.06 & 0 & 0 & 1 & \checkmark \checkmark \\
K01074.01 & Kepler-762 b & 5921 & 4.01 & $-27\pm25$ & $240^{+112}_{-96} $ & $> 548$ & Maybe & 15.19 & 3.77 & 0 & 2560 & 1 &  \\
K01839.01 & Kepler-971 b & 5447 & 6.22 & $105\pm6$ & $305^{+96}_{-112} $ & $366^{+290}_{-164}$ & Yes & 3.93 & 9.59 & 0 & 128 & 1 &  \\
K01833.01 & Kepler-968 b & 4413 & 10.46 & $10\pm18$ & $329^{+104}_{-88} $ & $> 159$ & Yes & 1.85 & 3.69 & 0 & 0 & 1 & Theia-520 \\
K01833.03 & Kepler-968 c & 4413 & 10.46 & $10\pm18$ & $329^{+104}_{-88} $ & $> 159$ & Yes & 1.63 & 5.71 & 0 & 0 & 1 & Theia-520 \\
K01833.02 & Kepler-968 d & 4413 & 10.46 & $10\pm18$ & $329^{+104}_{-88} $ & $> 159$ & Yes & 2.28 & 7.68 & 4 & 0 & 1 & Theia-520 \\
K00775.02 & Kepler-52 b & 4164 & 11.85 & $22\pm18$ & $353^{+184}_{-88} $ & $> 100$ & Yes & 2.19 & 7.88 & 0 & 0 & 1 & Theia-520 \\
K00775.01 & Kepler-52 c & 4164 & 11.85 & $22\pm18$ & $353^{+184}_{-88} $ & $> 100$ & Yes & 2.04 & 16.38 & 0 & 0 & 1 & Theia-520 \\
K00775.03 & Kepler-52 d & 4164 & 11.85 & $22\pm18$ & $353^{+184}_{-88} $ & $> 100$ & Yes & 2.03 & 36.45 & 0 & 0 & 1 & Theia-520 \\
K02675.01 & Kepler-1312 b & 5584 & 6.13 & $86\pm4$ & $361^{+72}_{-112} $ & $642^{+617}_{-318}$ & Yes & 2.07 & 5.45 & 0 & 2048 & 1 & \checkmark \checkmark \\
K02675.02 & Kepler-1312 c & 5584 & 6.13 & $86\pm4$ & $361^{+72}_{-112} $ & $642^{+617}_{-318}$ & Yes & 0.94 & 1.12 & 0 & 2048 & 1 & \checkmark \checkmark \\
K04004.01 & Kepler-1933 b & 5576 & 6.21 & $85\pm3$ & $369^{+72}_{-112} $ & $642^{+603}_{-318}$ & Yes & 1.01 & 4.94 & 4 & 0 & 1 &  \\
K02174.03 & Kepler-1802 b & 4245 & 11.45 & -- & $377^{+200}_{-104} $ & -- & -- & 1.71 & 7.73 & 4 & 308 & 0 &  \\
K02174.02 & Kepler-1802 c & 4245 & 11.45 & -- & $377^{+200}_{-104} $ & -- & -- & 2.05 & 33.14 & 0 & 308 & 0 &  \\
K03935.01 & Kepler-1532 b & 5554 & 6.48 & $90\pm6$ & $401^{+64}_{-104} $ & $567^{+545}_{-278}$ & Yes & 1.26 & 1.09 & 0 & 0 & 1 & \checkmark \checkmark \\
K01801.01 & Kepler-955 b & 5221 & 7.5 & $79\pm4$ & $401^{+96}_{-136} $ & $536^{+481}_{-237}$ & Yes & 2.69 & 14.53 & 0 & 0 & 1 & \checkmark \checkmark \\
K01800.01 & Kepler-447 b & 5648 & 6.4 & $103\pm3$ & $417^{+64}_{-72} $ & $405^{+355}_{-203}$ & Yes & 18.49 & 7.79 & 4 & 1024 & 1 &  \\
K03370.02 & Kepler-1481 b & 4832 & 9.11 & $22\pm7$ & $409^{+120}_{-120} $ & $> 210$ & Yes & 1.09 & 5.94 & 0 & 0 & 1 & \checkmark \\
K04156.01 & Kepler-1943 b & 6002 & -- & $99\pm5$ & -- & $409^{+520}_{-254}$ & No & 1.29 & 4.85 & 4 & 518 & 1 &  \\
K00448.01 & Kepler-159 b & 4511 & 10.5 & $19\pm10$ & $417^{+160}_{-112} $ & $> 161$ & Yes & 2.3 & 10.14 & 0 & 2048 & 1 & \checkmark \\
K00448.02 & Kepler-159 c & 4511 & 10.5 & $19\pm10$ & $417^{+160}_{-112} $ & $> 161$ & Yes & 2.75 & 43.59 & 0 & 2048 & 1 & \checkmark \\
K00046.01 & Kepler-101 b & 5498 & -- & $100\pm5$ & -- & $419^{+349}_{-196}$ & No & 5.9 & 3.49 & 0 & 518 & 1 &  \\
K04169.01 & Kepler-1561 b & 5742 & 6.18 & $66\pm3$ & $425^{+72}_{-72} $ & $1409^{+1718}_{-788}$ & Maybe & 0.94 & 1.01 & 0 & 1024 & 1 & \checkmark \checkmark \\
K02708.01 & Kepler-1320 b & 4536 & 10.46 & $25\pm32$ & $425^{+168}_{-104} $ & $> 140$ & Yes & 1.39 & 0.87 & 0 & 0 & 1 & \checkmark \\
K00119.01 & Kepler-108 b & 5626 & -- & $100\pm5$ & -- & $438^{+402}_{-220}$ & No & 8.2 & 49.18 & 0 & 418 & 1 &  \\
K00119.02 & Kepler-108 c & 5626 & -- & $100\pm5$ & -- & $438^{+402}_{-220}$ & No & 7.78 & 190.32 & 4 & 418 & 1 &  \\
K00323.01 & Kepler-523 b & 5267 & 7.6 & $49\pm4$ & $441^{+88}_{-112} $ & $1824^{+2706}_{-1047}$ & Maybe & 1.9 & 5.84 & 0 & 0 & 1 & \checkmark \checkmark \\
K02115.01 & Kepler-67 b & 5126 & 10.39 & $83\pm10$ & $882^{+104}_{-120} $ & $458^{+451}_{-206}$ & Yes & 2.96 & 15.73 & 0 & 0 & 1 & \checkmark \checkmark \\
K00002.01 & Kepler-2 b & 6436 & -- & $83\pm4$ & -- & $485^{+924}_{-353}$ & -- & 16.42 & 2.2 & 0 & 7 & 1 &  \\
K03371.02 & Kepler-1482 b & 5330 & 7.7 & $52\pm3$ & $489^{+80}_{-88} $ & $1630^{+2285}_{-904}$ & Maybe & 1.0 & 12.25 & 0 & 640 & 1 &  \\
K03010.01 & Kepler-1410 b & 3808 & 14.19 & $-21\pm24$ & $489^{+545}_{-168} $ & $> 80$ & Yes & 1.39 & 60.87 & 0 & 0 & 1 & \checkmark \\
K03497.01 & Kepler-1512 b & 4894 & 9.33 & $14\pm8$ & $505^{+136}_{-112} $ & $> 295$ & Yes & 0.8 & 20.36 & 4 & 2690 & 1 &  \\
K03864.01 & Kepler-1698 b & 4866 & 9.49 & $2\pm8$ & $521^{+144}_{-120} $ & $> 358$ & Yes & 0.9 & 1.21 & 0 & 0 & 1 & \checkmark \\
K05447.02 & Kepler-1629 b & 5585 & 7.45 & -- & $529^{+64}_{-64} $ & -- & -- & 0.67 & 3.88 & 0 & 0 & 0 &  \\
K01779.01 & Kepler-318 b & 5799 & 7.09 & $65\pm3$ & $561^{+104}_{-72} $ & $1507^{+1876}_{-858}$ & Yes & 3.97 & 4.66 & 0 & 0 & 1 & \checkmark \checkmark \\
K01779.02 & Kepler-318 c & 5799 & 7.09 & $65\pm3$ & $561^{+104}_{-72} $ & $1507^{+1876}_{-858}$ & Yes & 3.1 & 11.82 & 4 & 0 & 1 &  \\
K03324.01 & Kepler-1469 b & 5356 & 8.15 & $-5\pm25$ & $561^{+72}_{-80} $ & $> 535$ & Yes & 2.53 & 21.86 & 0 & 0 & 1 & \checkmark \\
K04246.01 & Kepler-1576 b & 5794 & 7.09 & $17\pm8$ & $561^{+96}_{-72} $ & $> 613$ & Yes & 0.9 & 6.98 & 0 & 2048 & 1 & \checkmark \\
K02035.01 & Kepler-1066 b & 5847 & 7.0 & $60\pm4$ & $585^{+152}_{-80} $ & $2018^{+2616}_{-1187}$ & Maybe & 1.96 & 1.93 & 4 & 0 & 1 &  \\
K02084.01 & Kepler-1792 b & 4942 & 9.49 & $13\pm11$ & $585^{+152}_{-136} $ & $> 312$ & Yes & 2.15 & 4.2 & 0 & 0 & 1 & \checkmark \\
K03274.01 & Kepler-1451 b & 5675 & 7.82 & $42\pm4$ & $593^{+80}_{-64} $ & $> 316$ & Yes & 2.33 & 35.62 & 0 & 0 & 1 & \checkmark \\
K01615.01 & Kepler-908 b & 5670 & 7.88 & $67\pm4$ & $601^{+80}_{-64} $ & $1317^{+1607}_{-724}$ & Yes & 1.36 & 1.34 & 0 & 2560 & 1 &  \\
K02022.01 & Kepler-349 b & 5756 & 7.71 & $64\pm6$ & $617^{+96}_{-72} $ & $1686^{+2185}_{-976}$ & Yes & 1.99 & 5.93 & 0 & 0 & 1 & \checkmark \checkmark \\
K02022.02 & Kepler-349 c & 5756 & 7.71 & $64\pm6$ & $617^{+96}_{-72} $ & $1686^{+2185}_{-976}$ & Yes & 1.97 & 12.25 & 0 & 0 & 1 & \checkmark \checkmark \\
K00620.01 & Kepler-51 b & 5635 & 8.14 & $48\pm8$ & $625^{+72}_{-64} $ & $> 258$ & Yes & 6.62 & 45.16 & 0 & 0 & 1 & \checkmark \\
K00620.03 & Kepler-51 c & 5635 & 8.14 & $48\pm8$ & $625^{+72}_{-64} $ & $> 258$ & Yes & 5.49 & 85.32 & 4 & 0 & 1 &  \\
K00620.02 & Kepler-51 d & 5635 & 8.14 & $48\pm8$ & $625^{+72}_{-64} $ & $> 258$ & Yes & 9.04 & 130.18 & 0 & 0 & 1 & \checkmark \\
K02803.01 & Kepler-1877 b & 5506 & 8.37 & $3\pm12$ & $625^{+64}_{-64} $ & $> 726$ & Maybe & 0.55 & 2.38 & 0 & 0 & 1 & \checkmark \\
K00720.04 & Kepler-221 b & 5070 & 9.3 & $21\pm5$ & $633^{+120}_{-112} $ & $> 341$ & Yes & 1.51 & 2.8 & 0 & 0 & 1 & \checkmark \\
K00720.01 & Kepler-221 c & 5070 & 9.3 & $21\pm5$ & $633^{+120}_{-112} $ & $> 341$ & Yes & 2.86 & 5.69 & 0 & 0 & 1 & \checkmark \\
K00720.02 & Kepler-221 d & 5070 & 9.3 & $21\pm5$ & $633^{+120}_{-112} $ & $> 341$ & Yes & 2.57 & 10.04 & 0 & 0 & 1 & \checkmark \\
K00720.03 & Kepler-221 e & 5070 & 9.3 & $21\pm5$ & $633^{+120}_{-112} $ & $> 341$ & Yes & 2.58 & 18.37 & 4 & 0 & 1 &  \\
K03097.02 & Kepler-431 b & 6259 & 16.16 & $80\pm3$ & -- & $671^{+1092}_{-458}$ & -- & 0.93 & 6.8 & 4 & 2055 & 1 &  \\
K03097.03 & Kepler-431 c & 6259 & 16.16 & $80\pm3$ & -- & $671^{+1092}_{-458}$ & -- & 0.93 & 8.7 & 4 & 2055 & 1 &  \\
K03097.01 & Kepler-431 d & 6259 & 16.16 & $80\pm3$ & -- & $671^{+1092}_{-458}$ & -- & 1.08 & 11.92 & 4 & 2055 & 1 &  \\
K01982.01 & Kepler-1781 b & 5363 & 9.17 & -- & $705^{+80}_{-72} $ & -- & -- & 1.95 & 4.89 & 0 & 0 & 0 &  \\
K01835.02 & Kepler-326 b & 5142 & 9.56 & $8\pm8$ & $721^{+112}_{-96} $ & $> 506$ & Yes & 1.25 & 2.25 & 0 & 640 & 1 &  \\
K01835.01 & Kepler-326 c & 5142 & 9.56 & $8\pm8$ & $721^{+112}_{-96} $ & $> 506$ & Yes & 1.38 & 4.58 & 0 & 640 & 1 &  \\
K01835.03 & Kepler-326 d & 5142 & 9.56 & $8\pm8$ & $721^{+112}_{-96} $ & $> 506$ & Yes & 1.31 & 6.77 & 0 & 640 & 1 &  \\
K03375.01 & Kepler-1918 b & 5522 & 9.11 & -- & $721^{+72}_{-72} $ & -- & -- & 2.19 & 47.06 & 0 & 0 & 0 &  \\
K01797.01 & Kepler-954 b & 4736 & 10.68 & $4\pm11$ & $729^{+216}_{-184} $ & $> 270$ & Yes & 2.16 & 16.78 & 0 & 0 & 1 & \checkmark \\
K03681.01 & Kepler-1514 b & 5852 & 7.87 & $69\pm2$ & $745^{+345}_{-128} $ & $1302^{+1622}_{-754}$ & Yes & 11.94 & 217.83 & 0 & 516 & 1 &  \\
K03681.02 & Kepler-1514 c & 5852 & 7.87 & $69\pm2$ & $745^{+345}_{-128} $ & $1302^{+1622}_{-754}$ & Yes & 1.17 & 10.51 & 4 & 516 & 1 &  \\
K01821.01 & Kepler-963 b & 5383 & 9.42 & -- & $745^{+80}_{-72} $ & -- & -- & 2.64 & 9.98 & 0 & 2048 & 0 &  \\
K00647.01 & Kepler-634 b & 6272 & -- & $77\pm3$ & -- & $768^{+1250}_{-527}$ & -- & 2.13 & 5.17 & 0 & 7 & 1 &  \\
K02037.01 & Kepler-1995 b & 4746 & 10.8 & $9\pm21$ & $770^{+216}_{-192} $ & $> 223$ & Yes & 3.46 & 73.76 & 0 & 2048 & 1 & \checkmark \\
K01781.02 & Kepler-411 b & 4920 & 10.32 & $4\pm3$ & $778^{+160}_{-168} $ & $> 396$ & Yes & 2.2 & 3.01 & 4 & 0 & 1 &  \\
K01781.01 & Kepler-411 c & 4920 & 10.32 & $4\pm3$ & $778^{+160}_{-168} $ & $> 396$ & Yes & 3.47 & 7.83 & 0 & 0 & 1 & \checkmark \\
K01781.03 & Kepler-411 d & 4920 & 10.32 & $4\pm3$ & $778^{+160}_{-168} $ & $> 396$ & Yes & 3.46 & 58.02 & 4 & 0 & 1 &  \\
K00001.01 & Kepler-1 b & 5730 & -- & $81\pm3$ & -- & $785^{+845}_{-419}$ & No & 14.04 & 2.47 & 4 & 0 & 1 &  \\
 %\\
	%\hline
	K01546.01 & -- & $72^{+128}_{-48} $ & 11.92 & 0.92 & 1 & 0 & 1 \\
K03991.01 & -- & $72^{+88}_{-56} $ & 1.37 & 1.57 & 1 & 0 & 1 \\
K06188.01 & -- & $80^{+168}_{-56} $ & 2.75 & 1.65 & 0 & 0 & 1 \\
K06195.01 & -- & $88^{+208}_{-64} $ & 2.28 & 1.44 & 0 & 0 & 1 \\
K07449.01 & -- & $88^{+192}_{-64} $ & 20.43 & 1.32 & 0 & 1 & 0 \\
K07375.01 & -- & $104^{+232}_{-72} $ & 1.73 & 4.85 & 0 & 0 & 1 \\
K03936.01 & -- & $176^{+104}_{-64} $ & 2.69 & 141.61 & 0 & 0 & 1 \\
K06246.01 & -- & $224^{+112}_{-104} $ & 1.63 & 9.13 & 0 & 0 & 1 \\
K02680.01 & -- & $256^{+88}_{-96} $ & 32.13 & 14.41 & 0 & 1 & 1 \\
K03815.01 & -- & $280^{+96}_{-96} $ & 7.22 & 4.74 & 0 & 1 & 0 \\
K01839.02 & -- & $304^{+96}_{-112} $ & 4.93 & 80.41 & 1 & 1 & 1 \\
K03473.01 & -- & $360^{+128}_{-144} $ & 1.96 & 27.64 & 0 & 0 & 1 \\
K03339.01 & -- & $392^{+112}_{-128} $ & 2.55 & 23.13 & 0 & 1 & 0 \\
K00753.01 & -- & $464^{+72}_{-64} $ & 20.74 & 19.9 & 0 & 1 & 0 \\
K00883.01 & -- & $464^{+128}_{-128} $ & 13.95 & 2.69 & 0 & 0 & 1 \\
K02052.01 & -- & $480^{+264}_{-160} $ & 4.23 & 4.22 & 0 & 1 & 0 \\
K03371.01 & -- & $488^{+80}_{-88} $ & 1.32 & 58.13 & 1 & 0 & 1 \\
K03880.01 & -- & $488^{+272}_{-160} $ & 1.04 & 1.8 & 0 & 0 & 1 \\
K03864.02 & -- & $521^{+144}_{-128} $ & 0.89 & 18.26 & 0 & 0 & 1 \\
K01547.01 & -- & $585^{+80}_{-64} $ & 12.33 & 30.69 & 0 & 0 & 1 \\
K04504.01 & -- & $657^{+96}_{-88} $ & 1.57 & 4.13 & 0 & 0 & 0 \\
K02700.01 & -- & $673^{+264}_{-184} $ & 1.38 & 0.91 & 0 & 0 & 1 \\
K01758.01 & -- & $761^{+80}_{-72} $ & 3.52 & 5.6 & 0 & 0 & 0 \\

	\enddata
	\tablecomments{This table includes X, Y, Z... The machine-readable version,
		available online, includes additional columns for .... 
	}
\end{deluxetable}
%\endlongtable

\clearpage

\bibliographystyle{aasjournal}
\bibliography{bibliography}

%\appendix
%\section{Can you avoid appendices?}
%\label{app:interp}


\clearpage
\listofchanges

\end{document}
